%!TEX encoding = UTF-8 Unicode
%!TEX spellcheck = Italian

\documentclass[a4paper,11pt,oneside]{amsart}

\usepackage[italian]{babel}
%\usepackage[utf8]{inputenc}
\usepackage[T1]{fontenc}
\usepackage{microtype}

\usepackage{amsmath, latexsym, amsfonts, amssymb, amsthm, amscd}
%\usepackage{showkeys}
\usepackage{graphics,epsf,epsfig,psfrag}
\usepackage{perpage}
\usepackage{lastpage}
\usepackage{url}
\usepackage{color}
\usepackage{titlesec}
\usepackage{multicol}
\renewcommand{\multicolsep}{0.1em}


\usepackage[a4paper,scale={0.75,0.75},marginratio={1:1}]{geometry}

%%%%%%%%%%%%%%%%%%%%%%%%%%%%%%%%%%%%%%%%%%%%%%%%%%%%%%%%%%%%%%
%%%%%%%%%%%%%%%%% FANCY PAGE LAYOUT %%%%%%%%%%%%%%%%%%%%%%%%%%
%%%%%%%%%%%%%%%%%%%%%%%%%%%%%%%%%%%%%%%%%%%%%%%%%%%%%%%%%%%%%%

\usepackage{fancyhdr}

\pagestyle{fancy}
%\fancyhead[R]{\sf Curriculum Vit\ae}
%\fancyhead[L]{\sf Francesco Caravenna}
\fancyhead[R]{}
\fancyhead[L]{}
\cfoot{\small \thepage\ / \pageref{LastPage}}
\renewcommand{\headrulewidth}{0pt}
\renewcommand{\footrulewidth}{0pt}
\setlength{\footskip}{8mm}


\fancypagestyle{plain}{% 
%\fancyhead{\sf ~} % empty header
\fancyhf{}
\cfoot{\small \thepage\ / \pageref{LastPage}}
\renewcommand{\headrulewidth}{0pt}
\renewcommand{\footrulewidth}{0pt}
\setlength{\footskip}{6.2mm}
}


\frenchspacing

\setlength{\parindent}{0pt}

%\usepackage[T1]{fontenc}
%%\usepackage[latin1]{inputenc} %under Linux
%\usepackage[ansinew]{inputenc} %under Windows


%%%%%%%%%%%%%%%%%%%%%%%%%%%%%%%%%%%%%%%%%%%%%%%%%%%%%%%%%%%%%%
%%%%%%%%%%%%%%%%% LIST ENVIRONMENTS %%%%%%%%%%%%%%%%%%%%%%%%%%
%%%%%%%%%%%%%%%%%%%%%%%%%%%%%%%%%%%%%%%%%%%%%%%%%%%%%%%%%%%%%%


%\newenvironment{myenumerate}{%
%\renewcommand{\theenumi}{\arabic{enumi}}%
%\renewcommand{\labelenumi}{{\rm(\theenumi)}}%
%\begin{list}{\labelenumi}
%	{%
%	\setlength{\itemsep}{0.3em}%
%	\setlength{\topsep}{0.3em}%
%	\setlength{\partopsep}{0em}%
%	\setlength{\parsep}{0em}%
%	\setlength{\parskip}{0em}%
%	\setlength\leftmargin{1.5em}%
%	\setlength\labelwidth{1.1em}%
%	\setlength{\labelsep}{0.4em}%
%	\usecounter{enumi}%
%	}%
%	}%
%{\end{list}
%}
%
%
%\renewenvironment{enumerate}{
%\begin{myenumerate}}%
%{\end{myenumerate}} 

\newcounter{myenum}
\newenvironment{myenumerate}{%
  \begin{list}{[\arabic{myenum}]\rule{0pt}{1.1em}}%
 	{\setlength\leftmargin{2.4em}}%
%	\setlength\labelwidth{2.4em}%
     \setlength{\labelsep}{0.5em}
     \usecounter{myenum}}%
  {\end{list}} 


\newenvironment{myitemize}{%
\begin{list}{$\bullet$}% 
 	{%
	\setlength{\itemsep}{0.3em}%
	\setlength{\topsep}{0.3em}%
	\setlength{\partopsep}{0em}%
	\setlength{\parsep}{0em}%
	\setlength{\parskip}{0em}%
	\setlength\leftmargin{1.5em}%
	\setlength\labelwidth{1.1em}%
	\setlength{\labelsep}{0.4em}%
%	\usecounter{enumi}%
	}%
	}%
{\end{list}}


\renewenvironment{itemize}{
\begin{myitemize}}%
{\end{myitemize}} 


%%%%%%%%%%%%%%%%%%%%%%%%%%%%%%%%%%%%%%%%%%%%%%%%%%%%%%%%%%%%%%
%%%%%%%%%%%%%%%%%%%%%%%%% FOOTNOTES %%%%%%%%%%%%%%%%%%%%%%%%%%
%%%%%%%%%%%%%%%%%%%%%%%%%%%%%%%%%%%%%%%%%%%%%%%%%%%%%%%%%%%%%%

\MakePerPage[2]{footnote} % restarts footnote counter at every new page
\def\thefootnote{\fnsymbol{footnote}} % prints footnotes markers as symbols

%\long\def\symbolfootnote[#1]#2{\begingroup\def\thefootnote{\fnsymbol{footnote}}%
%\footnote[#1]{#2}\endgroup}
% for unnumbered footnotes, use \symbolfootnote[0]{text here}
% ... 1=*, 2=dagger, 3=doubledagger, etc.


%%%%%%%%%%%%%%%%%%%%%%%%%%%%%%%%%%%%%%%%%%%%%%%%%%%%%%%%%%%%%%
%%%%%%%%%%%%%%%%%%%%%%%%%%%%%%%%%%%%%%%%%%%%%%%%%%%%%%%%%%%%%%
%%%%%%%%%%%%%%%%%%%%%%%%%%%%%%%%%%%%%%%%%%%%%%%%%%%%%%%%%%%%%%

\definecolor{myred}{RGB}{176,16,16}
\definecolor{myblack}{RGB}{0,0,0}

\titleformat{\section}{\color{myred}\titlerule[0.7pt]\color{myblack}}
{\color{myred}\rule[-0.4em]{0.7pt}{1.55em}\color{myblack}}
{.35em}{\bfseries\sffamily\MakeUppercase}

\titlespacing{\section}{0pt}{*3}{*1.5}
%\titlespacing{\section}{-0.35em}{*3}{*1.5}


%\newcommand{\mysection}[1]{%
%  \ifhmode\par\fi
%  \removelastskip
%  \vskip 0.7em
%  \begingroup
%  \textcolor{myred}{\noindent\rule[-0.5ex]{\columnwidth}{1.2pt}\nobreak}\par
%  \noindent
%  \vskip -0.15em\nobreak
%  \leavevmode\textcolor{myred}{\rule[-0.35em]{1.2pt}{1.55em}}%
%  \MakeUppercase{\hskip 0.8ex \bfseries \sffamily #1}\par
%  \vskip 0.5em\nobreak
%  \endgroup%
%%  \textcolor{myred}{\noindent\hrulefill\nobreak}
%  \nopagebreak%
%  }
%
%\renewcommand{\section}{\mysection}

\newcommand\myskip{\vskip .175\baselineskip plus .175\baselineskip}
%\def\myskip{\smallskip}
\newcommand\bP{\mathbf{P}}
\newcommand\dd{\mathrm{d}}
\newcommand\cH{\mathcal{H}}
\newcommand\N{\mathbb{N}}

%%%%%%%%%%%%%%%%%%%%%%%%%%%%%%%%%%%%%%%%%%%%%%%%%%%%%%%%%%%%%%
%%%%%%%%%%%%%%%%%%%%%%%%%%%%%%%%%%%%%%%%%%%%%%%%%%%%%%%%%%%%%%
%%%%%%%%%%%%%%%%%%%%%%%%%%%%%%%%%%%%%%%%%%%%%%%%%%%%%%%%%%%%%%

\usepackage{pxfonts} % Sets Palatino as main font
\renewcommand{\sfdefault}{iwona} % Sets Iwona as sans-serif font
%\usepackage[euler-digits]{eulervm} % Alternative math fonts (Euler)

\begin{document}

\LARGE
{\huge\sffamily\bfseries Francesco Caravenna}\hfill 
{\normalsize\sffamily \raisebox{0.53em}{[aggiornato al \today]}}

{\sffamily Curriculum Vit\ae}

\vspace{1.5ex}

\normalsize

\section{Dati Personali}

\begin{itemize}

\item Nato il 15 marzo 1979 a Treviglio (BG)

\item \emph{Indirizzo:} via Cozzi 55,\ 20125 Milano,\ Italy\\
\phantom{\emph{Indirizzo:}} ufficio 3062, edificio U5\\
\phantom{\emph{Indirizzo:}} telefono +39 02 6448 5752

\item \textit{E-mail:} \texttt{francesco.caravenna@unimib.it}

\item \textit{Home page:} \url{http://www.matapp.unimib.it/~fcaraven/}

\item \textit{Conoscenze linguistiche:} Italiano (lingua madre), Inglese, Francese

\item \textit{Conoscenze informatiche:} C, R, HTML/CSS, \LaTeX

\end{itemize}


\section{Formazione}

\begin{itemize}
\item
(21 Ott 2005) \textit{Dottorato di Ricerca in Matematica},
Universit\`a degli Studi di Milano Bicocca 
e Universit\'e Paris 7 Denis Diderot (F). Relatori:
Giambattista Giacomin e Alberto Gandolfi.

\item
(19 Dic 2003)
\textit{Diploma di Licenza in Fisica} (voto: 70/70 e lode),
Scuola Normale Superiore di Pisa. Relatore: Sergio Caracciolo.

\item
(23 Set 2002)
\textit{Laurea in Fisica} (voto: 110/110 e lode), 
Universit\`a di Pisa. Relatore: Sergio Caracciolo.
\end{itemize}



\section{Posizioni Accademiche}

\begin{itemize}
\item (Nov 2010 -- oggi) \emph{Professore Associato di
Probabilit\`a e Statistica Matematica} (s.s.d. MAT/06),
Dipartimento di Matematica e Applicazioni, Universit\`a degli Studi di Milano-Bicocca.

\item (Ott 2006 -- Ott 2010) \textit{Ricercatore in Probabilit\`a
e Statistica Matematica} (s.s.d. MAT/06),
Dipartimento di Matematica Pura e Applicata, 
Universit\`a degli Studi di Padova.

\item (Ott 2005 -- Set 2006) \textit{Postdoc in Matematica} nel gruppo
di Erwin Bolthausen,
Institut f\"ur Mathematik, Universit\"at Z\"urich (CH).
\end{itemize}

\section{Riconoscimenti}

\begin{itemize}
\item Vincitore del ``\emph{Premio Guido Fubini 2011}'', 
assegnato dall'Istituto Superiore Mario Boella
a un giovane matematico che lavora in Italia nell'ambito
dei processi aleatori.
\end{itemize}

\section{Studenti di dottorato}

\begin{itemize}
\item Niccol\`o Torri, tesi \emph{``Localization and universality phenomena for
random polymers''} discussa il 18 settembre 2015,
Universit\'e Claude Bernarde Lyon 1 (F) in cotutela con
l'Universit\`a degli Studi di Milano-Bicocca
(supervisione congiunta con Fabio Lucio Toninelli).

%\noindent
%[Attualmente post-doc presso il Dipartimento
%di Matematica dell'Universit\`a di Nantes (F)]

\item Jacopo Corbetta, tesi \emph{``General smile asymptotics and a multiscaling
stochastic volatility model''} discussa il 4 marzo 2015,
Universit\`a degli Studi di Milano-Bicocca.

%\noindent
%[Attualmente post-doc presso CERMICS,
%\'Ecole des Ponts - ParisTech,
%Marne la Vall\'ee (F)]
\end{itemize}


\section{Finanziamenti}

\begin{itemize}
\item Coordinatore (P.I.) del Progetto di Ricerca 
\textit{``Modelli probabilistici per la meccanica statistica di polimeri, 
sistemi di particelle interagenti e applicazioni''} (CPDA082105/08) 
dell'Universit\`a degli Studi di Padova (2008) [finanziamento di 34\,000 euro]

\item Membro di progetti PRIN a tema probabilit\`a/meccanica statistica (anni 2004, 2006, 2009).
\end{itemize}

\section{Organizzazione di scuole e conferenze}

\begin{itemize}
\item Organizzazione della Winter School
\emph{``Recent Breakthroughs in Singular Stochastic PDEs''}
(Universit\`a di Milano-Bicocca, 2-6 febbraio 2015),
con mini-corsi di Massimiliano Gubinelli e Lorenzo Zambotti 
[in collaborazione con Federica Masiero e Gianmario Tessitore].

\item Membro del comitato organizzatore del
\emph{``XII Workshop on Quantitative Finance''} (Padova, 27-28 gennaio 2011).
\end{itemize}

\section{Soggiorni di Ricerca}

\begin{itemize}

\item Cinque mesi presso il \emph{Mathematisch Instituut}, \emph{Universiteit Leiden} (NL)
come visiting professor nell'ambito del progetto
ERC Advanced Grant VARIS, su invito di Frank den Hollander
(17 marzo - 11 maggio 2012; 1 maggio - 30 giugno 2013; 1-30 marzo 2014) 

\item Due mesi presso il 
\textit{Laboratoire de Probabilit\'es et Mod\`eles Al\'eatoires},
\textit{Universit\'es Paris 6 e Paris 7} (F), su invito di Giambattista Giacomin
e Lorenzo Zambotti  (5 ottobre - 13 dicembre 2009) .

\item Un mese presso l'\emph{Institut f\"ur Mathematik}, \emph{Technische Universit\"at Berlin} (D)
su invito di Jean-Dominique Deuschel (8 gennaio - 2 febbraio 2007) .

\item Diversi periodi brevi (una/due settimane) presso la \emph{University of Warwick} 
(16-21 giu 2014),
la \emph{National University of Singapore} (4-15 apr 2016, 27 gen - 8 feb 2014),
l'\emph{Universit\'e de Nantes} (18-22 ott 2010),
la \emph{Technische Universit\"at Berlin} (3-14 nov 2008), l'\emph{ENS Lyon} (16-20 giu 2008),
l'\textit{Universit\'e Paris 7} (17-21 lug 2006, 19-23 mag 2008),
l'\textit{Universit\"at Z\"urich} (28 apr - 2 mag 2008),
la \emph{Technische Universiteit Eindhoven},
centro \emph{Eurandom} (10-14 lug 2006, 22-26 ott 2007).
\end{itemize}

%\newpage





\section{Attivit\`a di Referee}

\begin{itemize}
\item Ho svolto attivit\`a di \textit{referee} per le seguenti riviste:
\textit{Probability Theory and Related Fields}, 
\textit{Annals of Probability},
\textit{Communications in Mathematical Physics},
\textit{Stochastic Processes and their Applications},
\textit{Electronic Journal of Probability},
\textit{Annales de l'Institut Henri Poincar\'e}, 
\textit{Statistics and Probability Letters},
\textit{Markov Processes and Related Fields},
\textit{Potential Analysis}.
\end{itemize}





%\newpage



\section{Interessi di ricerca}

%\begin{multicols}{2}

La mia attivit\`a di ricerca \`e incentrata sulla teoria della probabilit\`a 
e sulle sue applicazioni. Le mie linee di ricerca principali sono le seguenti:
\begin{enumerate}
\item[\bf 1.] Meccanica statistica,
in particolare modelli probabilistici di polimeri
\item[\bf 2.] Propriet\`a asintotiche di passeggiate aleatorie reali
\item[\bf 3.] Modelli probabilistici per serie finanziarie
\end{enumerate}

\smallskip
{\hskip 1.2em}Sono coautore di 23 articoli
pubblicati su riviste internazionali
quali \emph{J. Eur. Math. Soc. (JEMS)}, 
\emph{Probab. Theory Related Fields}, \emph{Ann. Probab.}, 
\emph{Commun. Math. Phys.}, 
\emph{Ann. Appl. Probab.}, 
\emph{Stochastic Process. Appl.}, 
\emph{Electron. J. Probab.}, 
\emph{Ann. Inst. H. Poincar\'e}.
%\emph{J. Stat. Phys.}, 
%\emph{Electron. Commun. Probab.},
%\emph{Adv. in Appl. Probab.}.



%\smallskip
%
%{\hskip 1.2em}Presento di seguito una descrizione dei risultati ottenuti.
%
%\medskip
%
%\textbf{1.} 
%%Sin dalla mia tesi di dottorato [\ref{thesis}], 
%La mia linea di ricerca principale
%\`e %stata 
%lo studio di modelli probabilistici in meccanica statistica,
%in particolare modelli di polimeri in interazione con l'ambiente, tipicamente
%basati su passeggiate aleatorie o su processi di
%rinnovo (passeggiate aleatorie
%con incrementi positivi).
%Tali modelli, introdotti nelle scienze applicate,
%ricevono molta attenzione nella letteratura matematica
%per la ricca fenomenologia che esibiscono: fenomeni di localizzazione,
%transizioni di fase, universalit\`a.
%
%\smallskip
%
%\textbf{1a.} Un modello a cui ho dedicato molta attenzione \`e il cosiddetto
%\emph{copolimero}, o polimero disomogeneo, in prossimit\`a di un'interfaccia %selettiva
%che separa due solventi. La disomogeneit\`a del polimero \`e
%codificata da una successione reale $\omega = (\omega_n)_{n\in\N}$,
%che descrive le affinit\`a dei monomeri (i costituenti del polimero) per i solventi,
%mentre l'intensit\`a e 
%l'asimmetria dell'interazione sono regolate da due parametri reali $\lambda$ e~$h$.
%\`E possibile mostrare che esiste una
%curva critica $h = h_c(\lambda)$ che separa una regione ``localizzata'' $h < h_c(\lambda)$,
%in cui le configurazioni tipiche del polimero restano vicine all'interfaccia tra i solventi,
%da una regione ``delocalizzata'' $h > h_c(\lambda)$,
%in cui le configurazioni tipiche del polimero sono 
%invece interne a uno dei due solventi.
%
%{\hskip 1.2em}Lo studio della curva critica $h_c(\lambda)$
%\`e uno dei problemi principali per questo modello. 
%Le sue propriet\`a dipendono
%dalla scelta della successione $\omega$ che codifica la disomogeneit\`a.
%Il caso pi\`u interessante \`e quello in cui $\omega$ \`e
%una realizzazione di un processo aleatorio (tipicamente una successione i.i.d.):
%modelli di questo tipo sono detti \emph{sistemi disordinati},
%o fortemente disomogenei, e la loro analisi
%\`e particolarmente difficile e affascinante. Una classe di modelli pi\`u
%trattabili \`e data da sistemi debolmente disomogenei, in cui $\omega$ \`e
%una successione periodica.
%%o anche omogenei, in cui $\omega$ \`e costante.
%
%%Anche il caso limite dei sistemi omogenei, in cui $\omega$ \`e costante,
%%\`e interessante, perch\'e spesso il modello diventa esattamente risolubile.
%
%\smallskip
%
%{\hskip 1.2em}Per il modello di copolimero disordinato,
%una panoramica dei risultati noti e dei problemi aperti
%(aggiornata al 2012) \`e fornita nel proceeding~[\ref{cgt}],
%scritto in collaborazione con Giambattista Giacomin e Fabio Lucio Toninelli.
%Il proceeding [\ref{bari}] fornisce un'introduzione
%accessibile al modello, cos\`i come [\ref{cdhp}], scritto in collaborazione con Frank den Hollander
%e Nicolas P\'etr\'elis in occasione di un corso da noi tenuto per la
%XIV Scuola Brasiliana di Probabilit\`a congiuntamente con la 
%Scuola Estiva del Clay Mathematics Institute, nell'agosto 2010.
%
%{\hskip 1.2em}Venendo agli articoli su rivista,
%in collaborazione con Giambattista Giacomin ho mostrato 
%in~[\ref{cg}] che la tecnica del \textit{constrained annealing},
%estensivamente utilizzata nella letteratura fisica, \`e inefficace per migliorare
%le stime dall'alto sulla curva critica $h_c(\lambda)$ del modello di copolimero disordinato.
%In collaborazione anche con Massimiliano
%Gubinelli, ho elaborato in~[\ref{cgg1}] un test statistico con stima esplicita per la probabilit\`a
%di errore, basato su disuguaglianze di concentrazione,
%che ci ha permesso di fornire evidenze numeriche schiaccianti contro una congettura
%sulla curva critica, proposta nella letteratura fisica da C. Monthus
%[Eur. Phys. J. B 13 (2000), 111-130].
%
%
%{\hskip 1.2em}Mi sono quindi interessato alle propriet\`a 
%della curva critica $h_c(\lambda)$ nel regime di debole accoppiamento $\lambda \to 0$.
%In collaborazione con Giambattista Giacomin, ho mostrato in~[\ref{cg2}]
%che in questo regime si ha un fenomeno notevole di \emph{universalit\`a}:
%una vasta classe di modelli di copolimero esibisce lo
%stesso comportamento asintotico $h_c(\lambda) \sim m \, \lambda$,
%dove la costante $m$ non dipende dai dettagli del modello e ammette una caratterizzazione
%in termini di un modello continuo, costruito a partire da un moto browniano.
%Sono note stime esplicite dal basso e dall'alto sulla costante $m$, ma il suo valore esatto
%non \`e noto in generale. 
%Un'eccezione \`e data da modelli di copolimero costruiti
%a partire da processi di rinnovo con media finita. In questo caso speciale, in collaborazione
%con Quentin Berger, Julien Poisat, Rongfeng Sun e Nikos Zygouras, siamo riusciti a determinare
%in~[\ref{bcpsz}] il valore esatto della costante~$m$. 
%
%{\hskip 1.2em}
%I risultati in~[\ref{bcpsz}] si applicano anche alla
%classe dei modelli di \emph{pinning},
%che descrivono un polimero in interazione con un'interfaccia lineare.
%Su questi modelli c'\`e stata un'attivit\`a di ricerca molto intensa nella letteratura matematica
%recente, che ha permesso di dimostrare rigorosamente diversi risultati congetturati dai fisici
%e di risolvere alcune questione aperte.
%
%{\hskip 1.2em}
%Ritornando al modello di copolimero basato su un processo di rinnovo con media infinita,
%che \`e il caso pi\`u interessante,
%T. Bodineau e G. Giacomin avevano proposto in [J. Stat. Phys. 117 (2004), 801-818]
%un modello semplificato,
%con la speranza che catturasse le propriet\`a cruciali del modello originale,
%in particolare il valore della costante $m$.
%In collaborazione con Erwin Bolthausen e B\'eatrice de Tili\`ere,
%usando tecniche di rinormalizzazione e \textit{coarse-graining},
%siamo riusciti a determinare in~[\ref{bcdt}] il punto 
%critico di questo modello semplificato, mostrando
%che il modello originale di copolimero \`e in realt\`a sostanzialmente
%pi\`u ricco.
%
%{\hskip 1.2em}
%Uno strumento importante per lo studio dei modelli di polimeri disordinati,
%applicabile sia a modelli di copolimero che di pinning, \`e la
%disuguaglianza di regolarizzazione (\emph{smoothing}) ottenuta da G. Giacomin e
%F.L. Toninelli [Commun. Math. Phys. 266 (2006), 1-16].
%In collaborazione con Frank den Hollander, ne abbiamo fornito in~[\ref{cdh}]
%un'estensione, rimuovendo ipotesi non necessarie sulla
%distribuzione del disordine e fornendo costanti asintoticamente ottimali. Tale estensione
%si \`e rivelata un ingrediente importante per ottenere i risultati in~[\ref{bcpsz}].
%
%\smallskip
%
%\textbf{1b.} L'interesse per fenomeni di \emph{universalit\`a}, in cui 
%una vasta classe di modelli microscopici manifesta lo stesso comportamento macroscopico,
%ha guidato in modo cruciale la mia attivit\`a di ricerca recente, portando ad
%alcuni dei risultati che ritengo pi\`u importanti. 
%
%{\hskip 1.2em}Insieme a Rongfeng Sun
%e Nikos Zygouras, abbiamo formulato in~[\ref{csz1}] un quadro generale che permette
%di ottenere il limite delle funzioni di partizione, nel regime di debole accoppiamento,
%per modelli in cui il disordine \`e ``rilevante'' (ossia una quantit\`a di disordine
%arbitrariamente piccola, ma fissata, cambia le propriet\`a macroscopiche
%del sistema, ad es.\ gli esponenti critici). 
%Questo approccio pu\`o essere
%applicato a una vasta classe di sistemi disordinati: oltre ai gi\`a menzionati
%modelli di copolimero e pinning, abbiamo considerato in~[\ref{csz1}] il modello di 
%\emph{polimero diretto
%in ambiente aleatorio} e il \emph{modello di Ising bidmensionale
%con campo esterno aleatorio},
%che hanno ricevuto una grande attenzione nella letteratura matematica e fisica.
%
%{\hskip 1.2em}Le tecniche usate in~[\ref{csz1}] rivestono un interesse generale.
%Forniamo infatti condizioni esplicite per la convergenza di
%espansioni in ``caos polinomiale'' verso il ``caos di Wiener'',
%basate su un \emph{principio di Lindeberg}
%(o principio di invarianza) che estende risultati precedenti di
%E.~Mossel, R.~O'Donnel e K.~Oleszkiewicz [Ann. Math. 171 (2010), 295-341].
%
%{\hskip 1.2em}Ottenere i limiti delle funzioni di partizione \`e un passo cruciale
%per ricavare il limite di scala completo del modello. In collaborazione con Rongfeng Sun
%e Nikos Zygouras, abbiamo mostrato in~[\ref{csz2}] che \`e possibile costruire
%una versione continua del modello di pinning disordinato, quando il disordine \`e rilevante,
%ottenuta come limite di scala di una vasta classe di modelli di pinning discreti, 
%nel regime di debole accoppiamento.
%Tale modello continuo, che \`e basato su un insieme 
%rigenerativo stabile della retta reale (l'insieme degli zeri
%di un processo di Bessel) e su un disordine browniano,
%ha propriet\`a sottili e notevoli. Ad esempio, 
%per quasi ogni realizzazione del disordine, la legge del modello \`e singolare
%rispetto alla legge dell'insieme rigenerativo. Di conseguenza, il modello continuo di pinning 
%non ammette una Hamiltoniana (rispetto all'insieme rigenerativo) e la sua
%stessa costruzione rigorosa \`e un problema delicato.
%%I limiti delle funzioni di partizione possono essere viste come funzioni di partizione continue.
%
%{\hskip 1.2em}
%Nella prepubblicazione~[\ref{ctt}],
%con Fabio Lucio Toninelli e con il nostro ex studente di dottorato
%Niccol\`o Torri, abbiamo messo in evidenza 
%il valore di universalit\`a del modello continuo di pinning,
%mostrando che esso fornisce il comportamento asintotico della curva critica (e dell'energia libera)
%per una vasta classe di modelli di pinning discreti, nel regime
%di debole accoppiamento. Si tratta di un'estensione altamente non banale
%dei risultati ottenuti in~[\ref{cg2}] per il copolimero, perch\'e le tecniche
%usate in~[\ref{cg2}] si basano sull'esistenza di una Hamiltoniana continua, che
%non esiste nel caso del pinning.
%
%{\hskip 1.2em}
%Come gi\`a detto, l'approccio in~[\ref{csz1}] si applica a modelli in cui il disordine \`e
%``rilevante''. L'estensione a modelli in cui il disordine \`e ``marginalmente rilevante'',
%che sono tipicamente caratterizzati da divergenze logaritmiche, \`e l'oggetto della
%prepubblicazione~[\ref{csz3}], in collaborazione con Rongfeng Sun
%e Nikos Zygouras. In questo lavoro mostriamo che per modelli di pinning o di polimero diretto 
%``marginalmente rilevanti'' si verifica una transizione 
%tra una fase di ``disordine forte'' e una di ``disordine debole'' nel regime di debole accoppiamento.
%Mostriamo inoltre che la funzione di partizione
%ha un limite in legge log-normale nella fase di ``disordine debole''.
%Questi risultati permettono di dire qualcosa sull'\emph{equazione 
%del calore stocastica bidimensionale}
%con rumore bianco spazio-temporale moltiplicativo,
%%Un problema aperto importante \`e dare un senso rigoroso a tale equazione,
%che \`e mal posta in senso classico e sfugge anche alle recenti teorie
%di M. Hairer [Invent. Math. 198 (2014), 269-504] e M. Gubinelli, P. Imkeller, N. Perkowski
%[Forum of Mathematics, Pi, 2015, no. 6].
%Regolarizzando il potenziale mediante convoluzione con un nucleo regolare
%e riscalandone l'intensit\`a,
%%e mandando il parametro di regolarizzazione a zero, 
%abbiamo mostrato in~[\ref{csz3}] che,  nella fase di ``disordine debole'',
%la soluzione ha distribuzioni marginali asintoticamente log-normali
%e presenta un'interessante struttura di correlazioni multi-scala.
%%Tale equazione non pu\`o essere trattata nemmeno con la recente
%%teoria delle ``Strutture di Regolarit\`a'' di Martin Hairer
%%[Invent. Math. 198 (2014), 269-504] n\'e con la teoria alternativa delle
%%``Distribuzioni Paracontrollate'' di Gubinelli, Imkeller e Perkowski
%%[Forum of Mathematics, Pi, 2015, no. 6].
%
%\smallskip
%
%\textbf{1c.} Ho ottenuto risultati anche per altri sistemi disordinati.
%Nell'articolo~[\ref{cgg2}],
%in collaborazione con Giambattista Giacomin e Massimiliano Gubinelli,
%dimostriamo un teorema limite centrale per modelli di polimeri semi-flessibili
%costruiti componendo rotazioni aleatorie. Nell'articolo~[\ref{cpp}], in collaborazione
%con Philippe Carmona e Nicolas P\'etr\'elis, dimostriamo un fenomeno di localizzazione
%forte per una versione discreta del modello parabolico di Anderson con disordine a code pesanti.
%Nella prepubblicazione~[\ref{cdhpp}], in collaborazione con Frank den Hollander, Nicolas
%P\'etr\'elis e Julien Poisat, consideriamo un modello 
%``annealed'' (in cui si fa un'opportuna media sul disordine)
%di polimero, in cui i monomeri hanno una ``carica'' e si possono attrarre o respingere.
%Per questo modello, forniamo un'analisi completa del diagramma di fase,
%mostrando che c'\`e una transizione
%di fase di primo ordine e ottenendo risultati dettagliati (grandi deviazioni,
%legge dei grandi numeri, teorema limite centrale) per le quantit\`a di interesse.
%%Lo strumento chiave \`e una rappresentazione della funzione di partizione,
%%basata su una formula del tipo Ray-Knight per il tempo locale della passeggiata
%%aleatoria semplice sugli interi.
%
%\smallskip
%
%\textbf{1d.} 
%%Tutti i risultati descritti finora riguardano sistemi disordinati (fortemente disomogenei). 
%Nella prima
%fase della mia attivit\`a di ricerca mi sono anche interessato
%all'analisi di sistemi \textit{debolmente
%disomogenei}, in cui la successione $\omega$ che codifica la disomgeneit\`a
%\`e periodica (o anche costante, nel qual caso il modello \`e detto omogeneo).
%Tali modelli sono tipicamente pi\`u trattabili rispetto a quelli disordinati
%e permettono pertanto di ottenere risultati pi\`u forti.
%
%{\hskip 1.2em}Nell'articolo [\ref{cgz2}], in collaborazione
%con Giambattista Giacomin e Lorenzo Zambotti, abbiamo mostrato che modelli omogenei
%di pinning e copolimero ammettono una rappresentazione esplicita della funzione di partizione,
%basata sulla teoria del rinnovo, che permette di ottenerne il comportamento asintotico
%preciso (non solo logaritmico) nel limite di volume infinito.
%Abbiamo poi mostrato in [\ref{cgz1}] e [\ref{cgz3}] che questi risultati
%possono essere estesi a modelli debolmente disomogenei,
%sfruttando la teoria del rinnovo markoviana. 
%L'asintotica precisa della funzione di partizione ci ha permesso di analizzare
%in dettaglio le propriet\`a del modello, ottenendo risultati fini sulle traiettorie quali
%i limiti di scala verso processi continui (quali moto browniano, moto browniano riflesso,
%meandro browniano) e i limiti di volume infinito.
%
%{\hskip 1.2em}Un approccio basato sulla teoria del rinnovo markoviana
%\`e anche alla base 
%degli articoli [\ref{cd1}] e [\ref{cd2}], in collaborazione con Jean-Dominique Deuschel.
%In questi lavori analizziamo modelli di pinning e di \emph{wetting} (in cui si
%aggiunge il vincolo di positivit\`a) con interazione di tipo laplaciano,
%noti anche come modelli di membrana, in dimensione (1+1). 
%I risultati principali ottenuti in [\ref{cd1}] riguardano l'esistenza di una
%transizione di fase non banale e la relativa regolarit\`a: in particolare, mostriamo che
%la transizione \`e di secondo ordine per il modello di pinning, ma di primo ordine
%per il modello di wetting. L'analisi del modello di pinning \`e approfondita
%in [\ref{cd2}], dove deriviamo i limiti di scala delle configurazioni tipiche del modello e
%forniamo una caratterizzazione per traiettorie del diagramma di fase. 
%Questi modelli con interazione laplaciana sono fortemente
%legati all'integrale di una passeggiata aleatoria e, pertanto, alcuni dei problemi
%e risultati ottenuti nell'analisi hanno un interesse di carattere generale,
%come testimoniato dalle numerose citazione ottenute da [\ref{cd1}]
%al di fuori della comunit\`a di meccanica statistica.
%
%{\hskip 1.2em}Nell'articolo [\ref{bc}], in collaborazione con Martin Borecki, 
%ho considerato modelli di pinning in dimensione (1+1) con interazione mista
%di tipo gradiente pi\`u laplaciano, mostrando che, a differenza
%del caso di interazione puramente laplaciana, la transizione di fase sparisce.
%
%{\hskip 1.2em}Infine, in collaborazione con Nicolas P\'etr\'elis, ho studiato
%in [\ref{cp1}] e~[\ref{cp2}] un modello omogeneo di polimero in interazione
%attrattiva o repulsiva con un ambiente costituito da una successione di interfacce equispaziate.
%Sebbene non vi siano transizioni di fase al variare dell'intensit\`a dell'interazione,
%ci sono transizioni in termini della distanza $T_N$
%tra le interfacce (dove $N$ \`e la lunghezza del polimero), che determina \textit{scaling}
%radicalmente diversi per le traiettorie tipiche.
%Pi\`u precsamente, nel caso attrattivo si ha una transizione per $T_N \approx \log N$,
%mentre nel caso repusivo ci sono due transizioni distinte, per $T_N \approx N^{1/3}$
%e per $T_N \approx N^{1/2}$.
%%rinnovo ha permesso di fornire 
%
%\medskip
%
%\textbf{2.} Lo studio delle propriet\`a asintotiche delle passeggiate aleatorie reali 
%\`e un argomento classico in teoria della 
%probabilit\`a. Motivato anche dalle applicazioni alla meccanica statistica, 
%in cui diversi modelli sono costruiti a partire da passeggiate aleatorie, mi sono 
%interessato al comportamento asintotico di una passeggiata aleatoria 
%condizionata a restare positiva (su un orizzonte temporale finito),
%dimostrando un teorema limite locale sotto l'ipotesi 
%che la passeggiata aleatoria sia nel dominio di 
%attrazione della legge normale~[\ref{llt}].
%Le tecniche alla base di questo risultato derivano dalla teoria delle fluttuazioni
%per passeggiate aleatorie reali.
%
%{\hskip 1.2em}In collaborazione con Lo\"ic Chaumont,
%abbiamo considerato passeggiate aleatorie nel dominio di attrazione di una legge stabile,
%dimostrando un principio di invarianza (teorema limite centrale funzionale)
%nel caso in cui la passeggiata aleatoria sia
%condizionata a restare positiva per tutto il tempo~[\ref{carcha}].
%Successivamente abbiamo esteso tale principio di invarianza
%a \emph{ponti} di passeggiate aleatorie condizionate
%a restare positive~[\ref{carcha2}]. Questi risultati 
%hanno gi\`a trovato applicazioni
%interessanti, ad esempio
%nello studio di mappe planari aleatorie, a opera di N.~Curien e J.-F.~Le Gall
%[Ann. Inst. H. Poincar\'e (to appear), arXiv:1506.01590],
%e nello studio della diffusione di Ferrari-Spohn,
%a opera di D.~Ioffe, S.~Shlosman e Y.~Velenik
%[Commun. Math. Phys. 336 (2015), 905-932].
%
%{\hskip 1.2em}Pi\`u recentemente, nella prepubblicazione~[\ref{strong}]
%ho ottenuto condizioni necessarie
%e sufficienti per il teorema del rinnovo locale,
%detto anche ``strong renewal theorem'', per processi di rinnovo 
%nel dominio di attrazione di una legge stabile,
%risolvendo un problema aperto che risale a un articolo classico di Garsia e Lamperti
%del 1963. Lo stesso risultato \`e stato ottenuto indipendentemente e simultaneamente
%da Ron Doney, uno dei massimi esperti del settore. Abbiamo deciso di unire
%i nostri sforzi e stiamo attualmente collaborando
%per estendere il risultato a passeggiate aleatorie (con incrementi non necessariamente positivi).
%
%\medskip
%
%\textbf{3.} Il mio interesse per la modellistica di serie finanziarie \`e pi\`u recente.
%In collaborazione con Alessandro Andreoli, Paolo Dai Pra e Gustavo Posta,
%in~[\ref{acdp}] abbiamo introdotto e studiato un modello a volatilit\`a stocastica
%per serie finanziarie, affiancando ai risultati teorici un'analisi numerica
%effettuata sulla serie storica del Dow Jones Industrial Average. A dispetto della sua semplicit\`a,
%il modello \`e in grado di riprodurre accuratamente diverse propriet\`a delle serie finanziarie,
%incluso il cosiddetto ``multi-scaling dei momenti''.
%
%{\hskip 1.2em}Le propriet\`a finanziarie di questo modello, in particolare il prezzaggio
%di opzioni, sono state studiate in collaborazione col mio ex dottorando Jacopo Corbetta.
%Nella prepubblicazione~[\ref{carcor2}] abbiamo ottenuto formule asintotiche
%per la superficie di volatilit\`a implicita, valide in regimi generali di maturit\`a brevi
%e/o strike estremi, che mostrano in particolare come il ``volatility smile'' del nostro modello
%sia estremamente pronunciato. Tali formule asintotiche si basano su stime 
%sulla coda della distribuzione dei \emph{return}, per le quali usiamo tecniche
%di grandi deviazioni.
%
%{\hskip 1.2em}Il legame tra l'asintotica della volatilit\`a implicita e le
%stime sulla coda dei return \`e molto generale, come abbiamo mostrato
%nella prepubblicazione~[\ref{carcor1}], estendendo risultati
%di S. Benaim e P. Friz [Math. Finance 19 (2009), 1-12]
%a regimi ``misti'', in cui volatilit\`a e strike possono variare simultaneamente.
%Questo ci ha permesso di ottenere, sempre in~[\ref{carcor1}],
%nuovi sviluppi asintotici
%per diversi modelli, tra cui il popolare modello di Merton.




\newpage

\section{Pubblicazioni}




%\smallskip
\medskip
{\large\sf \bfseries Articoli su rivista}
\smallskip

\input{../Publications/pub}

\smallskip
\medskip
{\large\sf \bfseries Proceedings}
\smallskip

\input{../Publications/proc}

\smallskip
\medskip
{\large\sf \bfseries Libro di testo}
\smallskip

\begin{itemize}
\item F. Caravenna, P. Dai Pra.
\textit{Probabilit\`a. Un'introduzione attraverso modelli e applicazioni.}
Springer-Verlag Italia, Milano (2013).
\end{itemize}


% EXTENDED
%
%\smallskip
%\medskip
%{\large\sf \bfseries Indicatori bibliometrici (rilevati il 19 febbraio 2016)}
%\smallskip
%
%
%\begin{itemize}
%\item \emph{MathSciNet}: 125 citazioni, \ h-index 7 \ (23 documenti considerati)
%
%\item \emph{Scopus}: 131 citazioni, \ h-index 8 \ (21 documenti considerati)
%
%\item \emph{Web of Sience}: 131 citazioni, \ h-index 8 \ (19 documenti considerati)
%
%\item \emph{Google Scholar}: 394 citazioni, \ h-index 12 \ (36 documenti considerati)
%\end{itemize}






%\begin{myenumerate}
%%\setcounter{myenum}{-1}
%
%%\item\label{thesis}
%%F. Caravenna,
%%\textit{Random walk models and probabilistic techniques 
%%for inhomogeneous polymer chains},
%%Ph.D. Thesis (2005), Universit\`a di Milano-Bicocca 
%%and Universit\'e Paris~7.
%%
%\item\label{llt}
%F. Caravenna,
%\textit{A local limit theorem for random walks conditioned to stay positive},
%Probab. Theory Related Fields \textbf{133} (2005), 508--530.
%
%\item\label{cg}
%F. Caravenna, G. Giacomin,
%\textit{On constrained annealed bounds for pinning and wetting models},
%Electron. Comm. in Probab. \textbf{10} (2005), 179--189.
%
%\item\label{cgg1}
%F. Caravenna, G. Giacomin, M. Gubinelli,
%\textit{A numerical approach to copolymers at selective interfaces},
%J. Stat. Phys. \textbf{122} (2006), 799--832.
%
%\item\label{cgz1}
%F. Caravenna, G. Giacomin, L. Zambotti,
%\textit{A renewal theory approach to periodic copolymers with adsorption},
%Ann. Appl. Probab. \textbf{17} (2007), 1362--1398.
%
%\item\label{cgz2}
%F. Caravenna, G. Giacomin, L. Zambotti,
%\textit{Sharp asymptotic behavior for wetting models in (1+1)-dimension},
%Electron. J. Probab. \textbf{11} (2006), 345--362.
%
%\item\label{cgz3}
%F. Caravenna, G. Giacomin, L. Zambotti,
%\textit{Infinite volume limits of polymer chains with periodic charges},
%Markov Process. Related Fields \textbf{13} (2007), 697--730.
%
%\item\label{carcha}
%F. Caravenna, L. Chaumont,
%\textit{Invariance principles for random walks conditioned to stay positive},
%Ann. Inst. H. Poincar\'e Probab. Statist. \textbf{44} (2008), 170--190.
%
%\item\label{bari}
%F. Caravenna,
%\textit{Modelli di polimeri e passeggiate aleatorie},
%Boll. Unione Mat. Ital. Serie IX, vol. I (2008), 559--571
%% (disponibile su \texttt{http://www.math.unipd.it/{\footnotesize $\sim$}fcaraven/pub.html})
%
%\item\label{cd1}
%F. Caravenna, J.-D. Deuschel,
%\textit{Pinning and wetting transition for (1+1)-dimensional fields with Laplacian interaction},
%Ann. Probab. 36 (2008), 2388--2433.
%% (disponibile su \texttt{arXiv.org}: 0703434v2 [math.PR])
%
%\item\label{bcdt}
%E. Bolthausen, F. Caravenna, B. de Tili\`ere,
%\textit{The quenched critical point of a diluted disordered polymer model},
%Stochastic Process. Appl. 119 (2009), 1479--1504.
%% (disponibile su \texttt{arXiv.org}: 0711.0141v1 [math.PR])
%
%\item\label{cd2}
%F. Caravenna, J.-D. Deuschel,
%\textit{Scaling limits of (1+1)-dimensional pinning models with Laplacian interaction},
%Ann. Probab. 37 (2009), 903--945.
%
%\item\label{cgg2}
%F. Caravenna, G. Giacomin, M. Gubinelli,
%\textit{Large scale behavior of semiflexible heteropolymers},
%Ann. Inst. H. Poincar\'e Probab. Statist. 46 (2010), 97--118.
%
%\item\label{cp1}
%F. Caravenna, N. P\'etr\'elis,
%\textit{A polymer in a multi-interface medium},
%Ann. Appl. Probab. 19 (2009), 1803--1839.
%
%\item\label{cp2}
%F. Caravenna, N. P\'etr\'elis,
%\textit{Depinning of a polymer in a multi-interface medium},
%Electron. J. Probab. 14 (2009), 2038--2067.
%
%\item\label{cg2}
%F. Caravenna, G. Giacomin,
%\textit{The weak coupling limit of disordered copolymer models},
%Ann. Probab. 38 (2010), 2322--2378.
%
%\item\label{bc}
%M. Borecki, F. Caravenna,
%\textit{Localization for (1+1)-dimensional pinning models with $(\nabla + \Delta)$ interaction},
%Electron. Commun. Probab. 15 (2010), 534--548.
%
%\item\label{acdp}
%A. Andreoli, F. Caravenna, P. Dai Pra, G. Posta,
%\textit{Scaling and multiscaling in financial series: a simple model},
%Adv. in Appl. Probab. 44 (2012), 1018--1051.
%
%\item\label{cgt}
%F. Caravenna, G. Giacomin, F. L. Toninelli,
%\textit{Copolymers at selective interfaces: settled issues and open problems},
%in: Probability in complex physical systems. In honour of Erwin Bolthausen and 
%J\"urgen G\"artner. Edited by J.-D. Deuschel, B. Gentz, W. K\"onig, M. von Renesse, 
%M. Scheutzow, U. Schmock.
%Springer Proceedings in Mathematics 11 (2012), 289--312. 
%
%\item\label{cpp}
%F. Caravenna, P. Carmona, N. P\'etr\'elis,
%\textit{The discrete-time parabolic Anderson model with heavy-tailed potential},
%Ann. Inst. H. Poincar\'e 48 (2012), 1049--1080.
%
%\item\label{cdhp}
%F. Caravenna, F. den Hollander and N. P\'etr\'elis,
%\textit{Lectures on Random Polymers},
%in: Probability and Statistical Physics in Two and more Dimensions. 
%Proceedings of the Clay Mathematics Institute Summer School and 
%XIV Brazilian School of Probability (Buzios, Brazil). 
%Edited by David Ellwood, Charles Newman, Vladas Sidoravicius and Wendelin Werner.
%Clay Mathematics Proceedings 15 (2012), 319--393. 
%
%\item\label{bcpsz}
%Q. Berger, F. Caravenna, J. Poisat, R. Sun, N. Zygouras,
%\textit{The critical curve of the random pinning and copolymer models at weak coupling},
%Commun. Math. Phys. 326 (2014), 507--530. 
%
%\item\label{carcha2}
%F. Caravenna, L. Chaumont,
%\textit{An invariance principle for random walk bridges conditioned to stay positive},
%Electron. J. Probab. 18 (2013), no. 60, 1--32. 
%
%\item\label{cdh}
%F. Caravenna, F. den Hollander,
%\textit{A general smoothing inequality for disordered polymers},
%Electron. Commun. Probab. 18 (2013), no. 76, 1--15. 
%
%\item\label{csz1}
%F. Caravenna, R. Sun, N. Zygouras,
%\textit{Polynomial chaos and scaling limits of disordered systems},
%J. Eur. Math. Soc. (JEMS), to appear.
%
%\item\label{csz2}
%F. Caravenna, R. Sun, N. Zygouras,
%\textit{The continuum disordered pinning model},
%Probab. Theory Related Fields, to appear.
%
%
%
%\end{myenumerate}



%\medskip
%{\large\sf \bfseries Preprint}
%\smallskip
%
%\begin{myenumerate}
%\setcounter{myenum}{25}
%\item\label{carcor1} F. Caravenna, J. Corbetta,
%\textit{General smile asymptotics with bounded maturity},
%preprint (2014), arXiv.org:1411.1624.
%
%\item\label{carcor2}  F. Caravenna, J. Corbetta,
%\textit{The asymptotic smile of a multiscaling stochastic volatility model},
%preprint (2015), arXiv.org:1501.03387.
%
%\item\label{ctt} F. Caravenna, F. L. Toninelli, N. Torri,
%\textit{Universality for the pinning model in the weak coupling regime},
%preprint (2015), arXiv.org:1505.04927.
%
%\item\label{strong} F. Caravenna,
%\textit{The strong renewal theorem},
%preprint (2015), arXiv.org:1507.07502.
%
%\item\label{cdhpp} F. Caravenna, F. den Hollander, N. P\'etr\'elis, J. Poisat,
%\textit{Annealed scaling for a charged polymer},
%preprint (2015), arXiv.org:1509.02204.
%
%\item\label{csz3} F. Caravenna, R. Sun, N. Zygouras,
%\textit{Universality in marginally relevant disordered systems},
%preprint (2015), arXiv.org:1510.06287.
%\end{myenumerate}



%\newpage

\section{Seminari}
\label{sec:seminari}

\smallskip
Ho tenuto numerosi seminari su invito a conferenze italiane e internazionali, oltre
a svariati altri seminari in universit\`a e centri di ricerca e ad alcuni
seminari di carattere divulgativo. 

% EXTENDED
%{\hskip 1.2em}Ho tenuto per due volte (2007 e 2015) una conferenza breve 
%su invito al Congresso dell'Unione
%Matematica Italiana; sono stato \emph{plenary speaker}
%all'Italian Meeting on Hyperbolic Equations (2013);
%ho tenuto il Colloquium del Dipartimento di Matematica 
%all'Universit\`a di Leiden (2014).


%\newpage

%\smallskip
\medskip
{\large\sf \bfseries Seminari su invito}
\smallskip

\begin{itemize}
\item (14 Giu 2016)
\emph{Scaling and universality in Probability},
Mathematics Colloquium of the University of Luxembourg (L).

\item (10 Set 2015)
\emph{Un'estensione multilineare del teorema limite centrale
e il fenomeno dell'universalit\`a per sistemi disordinati},
conferenza breve su invito,
XX Congresso dell'Unione Matematica Italiana, Siena.

\item (31 Ago 2015)
\emph{Universality in marginally relevant disordered systems},
Conferenza ``Scaling Limits in Models of Statistical Mechanics'',
Mathematisches Forschungsinstitut Oberwolfach (D).

\item (5 Mag 2015)
\emph{Polynomial chaos and scaling limits of disordered systems},
Workshop on Stochastic Processes in Random Media,  Institute for Mathematical Sciences,
National University of Singapore.

\item (28 Lug 2014)
\emph{The continuum disordered pinning model},
SPA 2014 Conference, ``Random Polymers'' Invited Session, Buenos Aires (AR).

\item (6 Giu 2014)
\emph{Polynomial chaos and scaling limits of disordered systems},
Statistical Mechanics Conference, Universit\'e de Nantes (F).

\item (27 Apr 2014)
\emph{Polynomial chaos and scaling limits of disordered systems},
Mini-Workshop, NYU Abu Dhabi (UAE).

\item (13 Mar 2014)
\emph{Scaling and universality in Probability},
General Colloquium, Mathematisch Instituut,
Universiteit Leiden (NL).

\item (12 Set 2013)
\emph{Scaling limits and universality for random pinning models},
plenary speaker, 15th Italian Meeting on Hyperbolic Equations,
Universit\`a di Milano-Bicocca.

\item (9 Ago 2013)
\emph{Scaling limits and universality for random pinning models},
Workshop ``Universality and Scaling Limits in Probability and Statistical Mechanics'',
Hokkaido University, Sapporo (JP).

\item (21 Mar 2013)
\emph{Scaling limits and universality for random pinning models},
Workshop ``Analysis and Stochastics in Complex Physical Systems'', Universit\"at Leipzig (D).

%\item (10-12 Gen 2013)
%\emph{Random Polymers and Localization Strategies} [mini-corso],
%School of ``Random Polymers'', Eurandom, Eindhoven~(NL).
%
%\item (14-18 Mag 2012)
%\emph{Random Polymers and Localization Strategies} [mini-corso],
%Worshop ``Random Polymers and Related Topics'', Institute for Mathematical Sciences,
%National University of Singapore.
%
\item (15 Feb 2011)
\emph{The weak coupling limit of disordered copolymer models},
Workshop on Interacting Processes in Random Environments,
Fields Institute, Toronto (CDN).

\item (16 Ott 2010)
\emph{The weak coupling limit of disordered copolymer models},
Workshop ``Probabilistic Methods in Statistical Physics''
in occasione dei 65 anni di Erwin Bolthausen, Technische Universit\"at Berlin (D).

%\item (2-7 Ago 2010)
%\emph{Random Polymers} [esercitazioni per il corso tenuto da Frank den Hollander; 
%in collaborazione con Nicolas P\'etr\'elis],
%XIV Scuola Brasiliana di Probabilit\`a congiuntamente con la 
%Scuola Estiva del Clay Mathematics Institute
%``Probability and Statistical Physics in Two and more Dimensions'',
%B\'uzios -- Rio de Janeiro (BR).

\item (7 Giu 2010)
\emph{Scaling and multiscaling in financial indexes: a simple model},
Colloque ``M\'ecanique Statistique et Milieux Al\'eatoires'',
Universit\'e de Nantes~(F).

\item (11 Giu 2008)
\textit{Pinning and wetting transition for
(1+1)-dimensional fields with Laplacian interaction},
Conferenza ``Gradient Models and Elasticity'', University of Warwick (UK).

\item (6 Mar 2008)
\textit{The quenched critical point of a diluted disordered polymer model}, 
Workshop GREFI-MEFI 2008, CIRM, Marseille (F).

\item (26 Set 2007)
\textit{Modelli di polimeri e passeggiate aleatorie},
conferenza breve su invito,
XVIII Congresso dell'Unione Matematica Italiana, Bari.

\item (22 Giu 2007)
\textit{On the phase diagram of random copolymers at selective interfaces},
Conferenza ``Random Polymer Models'', Eurandom, Eindhoven~(NL).

\item (21 Feb 2007)
\textit{Pinning and wetting models with Laplacian interaction in (1+1)-di\-men\-sion},
Conferenza ``Polymer Models and Related Topics'',
Laboratoire J.A. Dieudonn\'e, Universit\'e de Nice ``Sophia Antipolis'' (F).

\item (4 Set 2006)
\textit{Pinning models with Laplacian interactions in (1+1)-dimension},
Conferenza ``Spatial Random Processes and Statistical Mechanics'',
Mathematisches Forschungsinstitut Oberwolfach (D).

\item (6 Giu 2006)
\textit{A renewal theory approach to weakly inhomogeneous polymer models},
Conferenza ``Hydrodynamic Limits and Particle Systems'',
Centro di Ricerca Matematica ``Ennio De Giorgi'', Pisa.
\end{itemize}


%\pagebreak

%\smallskip
\medskip
{\large\sf \bfseries Altri Seminari}
\smallskip

\begin{itemize}
\item (15 Giu 2016)
\emph{Universality in marginally relevant disordered systems},
University of Luxembourg (L).

\item (2 Giu 2016)
\emph{Universality in marginally relevant disordered systems},
University of Paris Diderot (F).

\item (19 Nov 2015)
\emph{Multi-linear central limit theorems and scaling limits of disordered systems},
Vienna Seminar in Mathematical Finance and Probability (A).

\item (19 Mar 2015)
\emph{Polynomial chaos and scaling limits of disordered systems},
S\'eminaire de Probabilit\'es, ENS Lyon (F).

\item (11 Ott 2013)
\emph{Polynomial chaos and scaling limits of disordered systems},
Universit\`a degli Studi di Padova.

\item (12 Lug 2013)
\emph{Scaling limits and universality for random pinning models},
Rhein-Main Kolloquium Stochastik, Johannes Gutenberg Universit\"at Mainz (D).

\item (13 Giu 2013)
\emph{Scaling limits and universality for random pinning models},
Most Informal Probability Seminar, Universiteit Leiden (NL).

\item (10 Giu 2013)
\emph{Scaling limits and universality for random pinning models},
Universit\'e d'Angers (F).

\item (27 Lug 2012)
\emph{Scaling and multiscaling in financial series: a simple model},
Dipartimento di Statistica e Metodi Quantitativi,
Universit\`a degli Studi di Milano-Bicocca.

\item (10 Lug 2012)
\emph{A random copolymer near a selective interface},
Sapienza Universit\`a di Roma.

\item (3 Apr 2012)
\emph{Bootstrap percolation on Galton Watson trees},
Most Informal Probability Seminar, Universiteit Leiden (NL).

\item (12 Mar 2012)
\emph{Scaling and multiscaling in financial series: a simple model},
Universit\`a degli Studi di Modena e Reggio Emilia.

\item (23 Set 2011)
\emph{Scaling and multiscaling in financial indexes: a simple model},
Universit\`a di Roma Tor Vergata.

\item (16 Dic 2010)
\emph{A polymer in a multi-interface medium},
Oberseminar Stochastics,
Universit\"at Bonn (D).

\item (2 Nov 2010)
\emph{Scaling and multiscaling in financial indexes: a simple model},
Universit\`a degli Studi di Milano-Bicocca.

\item (10 Giu 2010)
\emph{The weak coupling limit of disordered copolymer models},
University of Warwick~(UK).

\item (20 Mag 2010)
\emph{Scaling and multiscaling in financial indexes: a simple model},
Universit\`a degli Studi di Padova.

\item (1 Feb 2010)
\emph{Scaling and multiscaling in financial indexes: a simple model},
Universit\`a di Pavia.

\item (8 Dec 2009)
\emph{A polymer in a multi-interface medium},
S\'eminaire de Probabilit\'es,
Laboratoire de Probabilit\'es et Mod\`eles Al\'eatoires,
Universit\'es Paris 6 e Paris 7 (F).

\item (5 Nov 2009)
\emph{Large scale behavior of semiflexible heteropolymers},
Universit\'e de Nantes (F).

\item (21 Lug 2009)
\textit{A polymer in a multi-interface medium},
Universit\`a di Roma 3.

\item (12 Nov 2008)
\textit{A polymer in a multi-interface medium},
Berliner Kolloquium Wahr\-schein\-lich\-keits\-theo\-rie
(IRTG Seminar), Humboldt Universit\"at zu Berlin (D).

\item (19 Giu 2008)
\textit{Pinning and wetting transition for
(1+1)-dimensional fields with Laplacian interaction},
S\'eminaire de Probabilit\'es, ENS Lyon (F).

\item (30 Apr 2008)
\textit{Pinning and wetting transition for
(1+1)-dimensional fields with Laplacian interaction},
Seminar on Stochastic Processes, Universit\"at Z\"urich (CH).

\item (23 Ott 2007)
\textit{The quenched critical point of a diluted disordered polymer model},
Random Spatial Structures Seminar, Eurandom, Eindhoven (NL).

\item (17 Lug 2007)
\textit{The quenched critical point of a diluted disordered polymer model},
Scuola Estiva di Probabilit\`a di St.~Flour (F).

\item (24 Gen 2007)
\textit{Free energy lower bounds for random copolymers at selective interfaces},
Technische Universit\"at Berlin (D).

\item (13 Lug 2006)
\textit{Pinning models with Laplacian interactions in (1+1)-dimension},
Random Spatial Structures Seminar, Eurandom, Eindhoven (NL).

\item (17 Mar 2006)
\textit{A renewal theory approach to periodically inhomogeneous polymer models},
Seminaire de Probabilit\'es et Statistique,
Centre de Math\'ematiques et Informatique,
Universit\'e de Provence, Marseille (F).

\item (21 Dic 2005)
\textit{A local limit theorem and an invariance principle
for random walks conditioned to stay positive},
Seminar on Stochastic Processes, ETH Z\"urich (CH).

\item (16 Dic 2005)
\textit{A renewal theory approach to periodically inhomogeneous polymer models},
Berlin-Leipzig Seminar, Technische Universit\"at Berlin (D).

\item (19 Lug 2005)
\textit{A renewal theory approach to periodically inhomogeneous polymer models},
Scuola Estiva di Probabilit\`a di St.~Flour (F).

\item (27 Mag 2005)
\textit{Copolimeri in prossimit\`a di un'interfaccia selettiva},
Universit\`a di Milano-Bicocca.

\item (2 Giu 2004)
\textit{Un th\'eor\`eme limite local pour des marches al\'eatoires 
sous la contrainte d'\^etre positives},
Rencontres de Probabilit\'es, Universit\'e de Rouen (F).
\end{itemize}

%\smallskip
\medskip
{\large\sf \bfseries Seminari di carattere divulgativo}
\smallskip

\begin{itemize}
\item (12 Mar 2015) \emph{La matematica in universit\`a},
seminario per gli studenti delle scuole superiori, giornata di orientamento
``Privavera in Bicocca 2015''.

\item (14 Mar 2013) \emph{Ricerca in matematica},
seminario per gli studenti delle scuole superiori, giornata di orientamento
``Privavera in Bicocca 2013''.

\item (5 Nov 2008) \emph{The Banach-Tarski Paradox},
Technische Universit\"at Berlin (D).

\item (29 Giu 2006) \emph{What is\ldots the Banach-Tarski Paradox?},
Z\"urich Graduate Colloquium (CH).
\end{itemize}



%\newpage

\section{Attivit\`a Didattica}

\nopagebreak

A partire dal dottorato, ho tenuto esercitazioni e corsi
per le lauree triennali e magistrali in diverse universit\`a
presso l'Universit\`a di Milano-Bicocca, 
l'Universit\`a di Padova, il Politecnico di Milano e l'Universit\`a di Zurigo.
Ho inoltre tenuto diversi corsi e mini-corsi di livello pi\`u avanzato (dottorato, scuole, workshop).

%\smallskip
%
%{\hskip 1.2em}L'esperienza maturata insegnando il primo
%corso di probabilit\`a  della laurea triennale
%in matematica, insieme a Paolo Dai Pra,
%ci ha condotto a scrivere un libro di testo, pubblicato da Springer,
%calibrato specificamente sulle esigenze degli studenti di matematica:
%
%\begin{itemize}
%\item F. Caravenna, P. Dai Pra.
%\textit{Probabilit\`a. Un'introduzione attraverso modelli e applicazioni.}
%%UNITEXT - La matematica per il 3+2.
%Springer-Verlag Italia, Milano (2013).
%\end{itemize}

%\smallskip
%
%{\hskip 1.2em}Riporto di seguito l'elenco degli insegnamenti da me tenuti.


\medskip
{\large\sf \bfseries Corsi in scuole e workshop}
\smallskip
\label{sec:corsiscuole}

\begin{itemize}
\item (2 Giu 2016)
Minicorso su \emph{Polynomial chaos and scaling limits of disordered systems} [3h],
%[corso ($5 \times 1.5$ ore)],
Group de travail ``M\'ecanique statistique, syst\`emes de particules'',
Universit\'e Paris Diderot (F).

\item (7 - 11 Mar 2016)
Corso su \emph{Polynomial chaos and scaling limits of disordered systems} [3~x~1.5h],
%[corso ($5 \times 1.5$ ore)],
Workshop YEP XIII, Eurandom, TU Eindhoven (NL).

\item (28 Set - 2 Ott 2015)
Corso su \emph{Polynomial chaos and scaling limits of disordered systems} [5~x~1.5h],
%[corso ($5 \times 1.5$ ore)],
Berlin-Potsdam Summer School 2015, Levico Terme (I).

\item (10-12 Gen 2013)
Mini-corso \emph{Random Polymers and Localization Strategies} [3~x~2h],
Scuola ``Random Polymers'', Eurandom, Eindhoven~(NL),
10-12 Gennaio 2013.

\item (14-18 Mag 2012)
Mini-corso \emph{Random Polymers and Localization Strategies} [3~x~2h + 1h],
Worshop ``Random Polymers and Related Topics'', Institute for Mathematical Sciences,
National University of Singapore, 14-18 Maggio 2012.

\item (6-10 Set 2011)
Mini-corso \emph{Lo strano comportamento di una passeggiata casuale} [1.5h + 2h],
Summer school matematica ``Alfa Class'', Progetto Diderot - Fondazione CRT,
Solonghello.

\item (2-7 Ago 2010)
Esercitazioni per il corso \emph{Random Polymers} 
[5x1h esercitazioni, in collaborazione con Nicolas P\'etr\'elis;
corso tenuto da Frank den Hollander],
XIV Scuola Brasiliana di Probabilit\`a congiuntamente con la 
Scuola Estiva del Clay Mathematics Institute
``Probability and Statistical Physics in Two and more Dimensions'',
B\'uzios -- Rio de Janeiro (BR).
\end{itemize}

%\pagebreak

\medskip
{\large\sf \bfseries Corsi per il dottorato}
\smallskip

\begin{itemize}
\item (a.a. 2013/14)
Mini-corso \textit{Random Graphs and Complex Networks},
Dottorato di Ricerca in Matematica, Universit\`a
di Milano, Milano-Bicocca, Pavia e Politecnico di Milano (corso interdottorato)
[in collaborazione con Federico Bassetti].

\item (a.a. 2009/10) Mini-corso \textit{Random Graphs and Complex Networks},
Dottorato di Ricerca in Matematica, Universit\`a degli Studi di Padova
[in collaborazione con Paolo Dai Pra].

\item (a.a. 2008/09)
Mini-corso \textit{Processi di punto di Poisson e applicazioni},
Dottorato di Ricerca in Matematica, Universit\`a degli Studi di Padova.

\item (a.a. 2006/07)
Mini-corso \textit{Modelli di polimeri e passeggiate aleatorie},
Dottorato di Ricerca in Matematica, Universit\`a degli Studi di Padova.
\end{itemize}



\medskip
{\large\sf \bfseries Corsi ed esercitazioni per la laurea triennale e magistrale}

\medskip

\emph{Presso l'Universit\`a degli Studi di Milano-Bicocca}

%\smallskip
\begin{itemize}
\item (a.a. 2015/16)
Corso di \textit{Processi Stocastici},
Laurea Magistrale in Matematica
[in collaborazione con Gianmario Tessitore].

\item (a.a. 2015/16)
Corso di \textit{Calcolo delle Probabilit\`a},
Laurea in Matematica.

\item (a.a. 2014/15)
Corso di \textit{Processi Stocastici},
Laurea Magistrale in Matematica
[in collaborazione con Gianmario Tessitore].

\item (a.a. 2014/15)
Corso ed esercitazioni di \textit{Calcolo delle Probabilit\`a},
Laurea in Matematica.

\item (a.a. 2013/14)
Corso di \textit{Processi Stocastici},
Laurea Magistrale in Matematica
[in collaborazione con Gianmario Tessitore].

\item (a.a. 2013/14)
Corso ed esercitazioni di \textit{Calcolo delle Probabilit\`a},
Laurea in Matematica.

\item (a.a. 2012/13)
Corso di \textit{Processi Stocastici},
Laurea Magistrale in Matematica
[in collaborazione con Gianmario Tessitore].

\item (a.a. 2012/13)
Corso ed esercitazioni di \textit{Calcolo delle Probabilit\`a},
Laurea in Matematica.

\item (a.a. 2011/12)
Corso di \textit{Processi Stocastici},
Laurea Magistrale in Matematica
[in collaborazione con Gianmario Tessitore].

\item (a.a. 2011/12)
Corso ed esercitazioni di \textit{Calcolo delle Probabilit\`a},
Laurea in Matematica.

\item (a.a. 2010/11)
Corso di \textit{Processi Stocastici},
Laurea Magistrale in Matematica
[in collaborazione con Gianmario Tessitore].

\item (a.a. 2004/05) Esercitazioni di \textit{Processi stocastici},
Laurea in Matematica
[docente: Gianmario Tessitore].

\item (a.a. 2003/04) Esercitazioni di \textit{Processi stocastici},
Laurea in Matematica
[docente: Daniela Bertacchi].

\item (a.a. 2003/04)
Esercitazioni di \textit{Calcolo delle probabilit\`a e statistica matematica},
Laurea in Matematica
[docente: Daniela Bertacchi].
\end{itemize}

\smallskip
\emph{Presso l'Universit\`a degli Studi di Padova}
%\smallskip

\begin{itemize}
\item (a.a. 2010/11)
Corso di \textit{Analisi Stocastica},
Laurea Magistrale in Matematica
[in collaborazione con David Barbato].

\item (a.a. 2009/10) Esercitazioni di \textit{Statistica}, Laurea in Biologia Molecolare,
[docente: Paolo Dai Pra].

\item (a.a. 2009/10)
Corso di \textit{Analisi Stocastica},
Laurea Magistrale in Matematica.

\item (a.a. 2008/09)
Esercitazioni di \textit{Probabilit\`a e Statistica},
Laurea in Matematica
[docente: Paolo Dai Pra].

\item (a.a. 2008/09)
Corso di \textit{Analisi Stocastica},
Laurea Magistrale in Matematica.

\item (a.a. 2007/08)
Corso di \textit{Matematica C} (probabilit\`a e analisi),
Laurea in Ingegneria Biomedica
[in collaborazione con Caterina Sartori].

\item (a.a. 2007/08)
Esercitazioni di \textit{Probabilit\`a e Statistica},
Laurea in Matematica
[docente: Paolo Dai Pra].

\item (a.a. 2007/08)
Esercitazioni di \textit{Metodi statistici per la biologia},
Laurea in Biologia
[docente: Paolo Dai Pra].

\item (a.a. 2006/07)
Esercitazioni di \textit{Metodi statistici per la biologia},
Laurea in Biologia [docente: Paolo Dai Pra].

\item (a.a. 2006/07)
Esercitazioni di \textit{Metodi statistici per la biologia},
Laurea in Biologia Molecolare [docente: Paolo Dai Pra].
\end{itemize}

\smallskip
\emph{Presso l'Universit\"at Z\"urich}
%\smallskip

\begin{itemize}
\item (a.a. 2005/06) Esercitazioni di \textit{Algebra lineare II},
Laurea in Matematica [docente: Erwin Bol\-thau\-sen].

\item (a.a. 2005/06) Esercitazioni di \textit{Algebra lineare I},
Laurea in Matematica [docente: Erwin Bol\-thau\-sen].
\end{itemize}

%\newpage

\smallskip
\emph{Presso il Politecnico di Milano}
%\smallskip

\begin{itemize}

\item (a.a. 2002/03) Esercitazioni on line (``Sessioni Live'') di
\textit{Calcolo delle probabilit\`a e statistica},
Laurea in Ingegneria Informatica Online [docente: Elio Piazza].

\item (a.a. 2002/03) \textit{Laboratorio numerico di statistica},
Laurea in Ingegneria Gestionale [docente: Elio Piazza].
\end{itemize}

%\newpage

% EXTENDED
%
%\section{Supervisione di tesi di laurea}
%\label{sec:tesi}
%
%Dal 2008 a oggi sono stato relatore di
%20 tesi di laurea triennale e 7 tesi di laurea magistrale 
%(oltre a 2 tesi di dottorato) in matematica.
%Cinque studenti brillanti, di cui sono stato relatore di laurea
%magistrale, hanno proseguito gli studi
%con un dottorato in prestigiose sedi estere:
%\begin{itemize}
%\item[-] Alessandro Garavaglia presso la Technische Universiteit Eindhoven (NL);
%
%\item[-] Marco Furlan presso l'Universit\'e Paris Dauphine (F);
% 
%\item[-] Elia Bisi presso la University of Warwick (UK);
%
%\item[-] Niccol\`o Torri presso l'Universit\'e Lyon 1 (F),
%in cotutela con l'Universit\`a degli Studi di Milano-Bicocca;
%
%\item[-] Enrico Zanardo presso la Columbia University di New York (US)
%[dottorato in economia].
%\end{itemize}
%
%%\smallskip
%
%%{\hskip 1.2em}Riporto di seguito l'elenco delle tesi di cui sono stato relatore.
%
%%\medskip
%%{\large\sf \bfseries Studenti di dottorato}
%%\smallskip
%%\label{sec:dottorato}
%%
%%\begin{itemize}
%%\item Niccol\`o Torri, tesi \emph{``Localization and universality phenomena for
%%random polymers''} discussa il 18 settembre 2015,
%%Universit\'e Claude Bernarde Lyon 1 (F) in cotutela con
%%l'Universit\`a degli Studi di Milano-Bicocca
%%(supervisione congiunta con Fabio Lucio Toninelli).
%%
%%\noindent
%%[Attualmente post-doc presso il Dipartimento
%%di Matematica dell'Universit\`a di Nantes (F)]
%%
%%\item Jacopo Corbetta, tesi \emph{``General smile asymptotics and a multiscaling
%%stochastic volatility model''} discussa il 4 marzo 2015,
%%Universit\`a degli Studi di Milano-Bicocca.
%%
%%\noindent
%%[Attualmente post-doc presso CERMICS,
%%\'Ecole des Ponts - ParisTech,
%%Marne la Vall\'ee (F)]
%%\end{itemize}
%
%
%%\newpage
%
%\medskip
%{\large\sf \bfseries Tesi di laurea magistrale}
%\smallskip
%%\section{Supervisione tesi di laurea magistrali}
%%
%\nopagebreak
%
%\begin{itemize}
%\item \emph{Diameter in ultra-small scale-free random graphs},
%studente Alessandro Garavaglia, laurea magistrale in matematica,
%Universit\`a degli Studi di Milano-Bicocca, a.a. 2013/14
%(direzione congiunta con Remco van der Hofstad, TU Eindhoven).
%
%\item \emph{Stochastic Navier-Stokes equation in 3 dimensions},
%studente Marco Furlan, laurea magistrale in matematica,
%Universit\`a degli Studi di Milano-Bicocca, a.a. 2012/13
%(direzione congiunta con Gianmario Tessitore e con Massimiliano
%Gubinelli, Universit\'e Paris Dauphine).
%
%\item \emph{Large Deviations},
%studente Elia Bisi, laurea magistrale in matematica,
%Universit\`a degli Studi di Milano-Bicocca, a.a. 2012/13.
%
%\item \emph{Il grafo aleatorio di Erd\"os-Renyi},
%studentessa Alice Sala, laurea magistrale in matematica,
%Universit\`a degli Studi di Milano-Bicocca, a.a. 2011/12.
%
%\item \emph{Localization phenomena for linear chains with correlated disorder},
%studente Niccol\`o Torri, laurea magistrale in matematica,
%Universit\`a degli Studi di Milano-Bicocca, a.a. 2011/12.
%
%\item \emph{Kingman's coalescent and some applications to population genetics},
%studente Vanni Rovera, laurea magistrale in matematica,
%Universit\`a degli Studi di Milano-Bicocca, a.a. 2010/11.
%
%\item \emph{Bootstrap percolation on random trees},
%studente Enrico Zanardo, laurea magistrale in matematica,
%Universit\`a degli Studi di Padova, a.a. 2010/11
%(direzione congiunta con Marek Biskup, UCLA).
%\end{itemize}
%
%%\newpage
%
%\medskip
%{\large\sf \bfseries Tesi di laurea triennale}
%\smallskip
%%\section{Supervisione tesi di laurea triennali}
%%
%\nopagebreak
%
%\begin{itemize}
%\item \emph{Il modello di Ising},
%studentessa Federica Sabia, laurea triennale in matematica,
%Universit\`a degli Studi di Milano-Bicocca, a.a. 2014/15.
%
%\item \emph{Condensazione di Bose-Einstein},
%studentessa Marta-Clarissa Arrighi, laurea triennale in matematica,
%Universit\`a degli Studi di Milano-Bicocca, a.a. 2014/15.
%
%\item \emph{Disuguaglianze di concentrazione},
%studentessa Michela Montrasio, laurea triennale in matematica,
%Universit\`a degli Studi di Milano-Bicocca, a.a. 2014/15.
%
%\item \emph{Convergenza all'equilibrio per catene di Markov},
%studente Lorenzo Madasi, laurea triennale in matematica,
%Universit\`a degli Studi di Milano-Bicocca, a.a. 2013/14.
%
%\item \emph{Character Theory and Random Walks on Finite Groups},
%studentessa Marta Maggioni, laurea triennale in matematica,
%Universit\`a degli Studi di Milano-Bicocca, a.a. 2013/14
%(direzione congiunta con Thomas Weigel).
%
%\item \emph{Catene di Markov a tempo continuo},
%studente Davide Cortesi, laurea triennale in matematica,
%Universit\`a degli Studi di Milano-Bicocca, a.a. 2013/14.
%
%\item \emph{Modellizzare gli eventi estremi},
%studente Jacopo Zanzi, laurea triennale in matematica,
%Universit\`a degli Studi di Milano-Bicocca, a.a. 2013/14.
%
%\item \emph{La convergenza debole e la misura di Wiener},
%studente Marco Moraschini, laurea triennale in matematica,
%Universit\`a degli Studi di Milano-Bicocca, a.a. 2012/13.
%
%\item \emph{Quanto pu\`o sporgere un muro a secco?},
%studentessa Michela Testa, laurea triennale in matematica,
%Universit\`a degli Studi di Milano-Bicocca, a.a. 2012/13.
%
%\item \emph{Percolazione},
%studentessa Carola Corti, laurea triennale in matematica,
%Universit\`a degli Studi di Milano-Bicocca, a.a. 2012/13.
%
%%\item \emph{Planar flows from the point of view of complex analysis},
%%studente Nikita Simonov, laurea triennale in matematica,
%%Universit\`a degli Studi di Milano-Bicocca, a.a. 2012/13
%%(correlatore; relatore Pietro Poggi-Corradini).
%
%\item \emph{Aspetti combinatori della passeggiata aleatoria semplice},
%studentessa Sara Locatelli, laurea triennale in matematica,
%Universit\`a degli Studi di Milano-Bicocca, a.a. 2011/12.
%
%\item \emph{Cammini autoevitanti sul reticolo esagonale},
%studentessa Claudia Piccinini, laurea triennale in matematica,
%Universit\`a degli Studi di Milano-Bicocca, a.a. 2011/12.
%
%%\item \emph{Descrizione probabilistica delle reazioni chimiche},
%%studentessa Giulia Mezzadri, laurea triennale in matematica,
%%Universit\`a degli Studi di Milano-Bicocca, a.a. 2011/12 (correlatore).
%
%\item \emph{Propriet\`a asintotiche delle catene di Markov},
%studentessa Krizia Giglione, laurea triennale in matematica,
%Universit\`a degli Studi di Milano-Bicocca, a.a. 2011/12.
%
%\item \emph{Il teorema ergodico},
%studente Aniello Cerullo, laurea triennale in matematica,
%Universit\`a degli Studi di Milano-Bicocca, a.a. 2011/12.
%
%\item \emph{Alcuni risultati nella teoria delle grandi deviazioni},
%studentessa Ga\"elle Iris Touwaide, laurea triennale in matematica,
%Universit\`a degli Studi di Milano-Bicocca, a.a. 2010/11.
%
%\item \emph{Opzioni americane e problemi di arresto ottimale},
%studente Federico Favero, laurea triennale in matematica,
%Universit\`a degli Studi di Padova, a.a. 2010/11.
%
%\item \emph{Il modello di Curie-Weiss e le grandi deviazioni},
%studentessa Valentina Favero, laurea triennale in matematica,
%Universit\`a degli Studi di Padova, a.a. 2009/10.
%
%\item \emph{Il teorema ergodico},
%studente Mauro Timini, laurea triennale in matematica,
%Universit\`a degli Studi di Padova, a.a. 2009/10.
%
%\item \emph{Il processo di ramificazione di Galton-Watson}, studentessa
%Roberta Maroni, laurea triennale in matematica,
%Universit\`a degli Studi di Padova, a.a. 2008/09.
%
%\item \emph{Modelli di polimeri, passeggiate aleatorie e processi di rinnovo},
%studentessa Diana Frizziero, laurea triennale in matematica,
%Universit\`a degli Studi di Padova, a.a. 2008/09.
%\end{itemize}
%
%%\newpage
%
%
%
%
%\section{Attivit\`a gestionali, organizzative e di servizio}
%
%%\bigskip
%
%\medskip
%
%\emph{Presso l'Universit\`a degli Studi di Milano-Bicocca}
%\smallskip
%
%\begin{itemize}
%\item Membro del Collegio dei Docenti del
%Dottorato
%in Matematica Pura e Applicata, 
%%Universit\`a degli Studi  di Milano-Bicocca, 
%cicli XXIX (anno di inizio 2013) e
%XXX (anno di inizio 2014);
%
%\item Membro del Collegio dei Docenti del
% Dottorato in Matematica consortile con
%l'Universit\`a di Pavia (sede amministrativa) e l'Indam, ciclo XXXI (anno di inizio 2015).
%
%%\item Assicuratore di qualit\`a?
%
%\item Membro della commissione 
%per l'attribuzione degli assegni di ricerca %(junior e senior)
%del Dipartimento di Matematica
%e Applicazioni (anni 2014 e 2015).
%
%\item Membro della commissione per l'attribuzione delle
%esercitazioni e dei tutorati assegnati tramite bando (anno 2015).
%
%\item 
%%Seminari a tema ``La matematica in universit\`a''
%%per studenti delle scuole superiori,
%Partecipazione alle giornate di orientamento ``Primavera in Bicocca''
%2013 e 2015.
%%(14 marzo 2013; 12 marzo 2015)
%
%
%\item Sorveglianza ai test d'ingresso per i corsi di laurea della Scuola di Scienze (anno 2014).
%
%%\item Assicuratore di qualit\`a
%
%\end{itemize}
%
%\smallskip
%\emph{Presso l'Universit\`a degli Studi di Padova}
%\smallskip
%
%\begin{itemize}
%\item Componente della Giunta del Dipartimento di Matematica
%Pura e Applicata
%%(una versione ristretta del Consiglio di Dipartimento) 
%per due anni (13 ottobre 2008 -- 31 ottobre 2010).
%
%\item Assistenza ai test di ammissione della Facolt\`a di Scienze mm. ff. nn.
%(anni 2007 e 2010).
%%(3 settembre 2007, 1 settembre 2010)
%\end{itemize}



%\bigskip
%
%\rule{\columnwidth}{.8pt}
%
%\bigskip
%
%Le dichiarazioni rese nel presente curriculum sono da ritenersi 
%rilasciate ai sensi degli artt. 46 e 47 del DPR n. 445/2000.
%
%\medskip
%
%Il presente curriculum, non contiene dati sensibili e 
%dati giudiziari di cui all'art. 4, comma 1, lettere d) ed e) del D.Lgs. 30.6.2003 n. 196.
%
%\medskip
%
%Il sottoscritto dichiara di essere consapevole che nel 
%rispetto delle regole di trasparenza previste dalla Legge e come stabilito dal bando di concorso, 
%i curricula di tutti candidati saranno pubblicati sul sito Web dell'Universit\`a 
%degli Studi di Milano \url{www.unimi.it/valcomp}.
%
%\bigskip
%\medskip
%\smallskip
%
%Data \qquad \boxed{\rm \ 17 \ gennaio \ 2016\ }
%\qquad\qquad \quad Luogo \qquad \boxed{\rm \ Cassano \ d'Adda \ (MI) 
%\hspace{3.1cm}}
%
%\bigskip
%\bigskip
%\bigskip
%
%\hspace{8.7cm} \rule[-.1cm]{6.88cm}{.8pt}
%
%\hspace{8.7cm} Firma


\end{document}
