%!TEX encoding = UTF-8 Unicode
%!TEX spellcheck = English

\documentclass[a4paper,11pt,oneside]{amsart}

%\usepackage[utf8]{inputenc}
\usepackage[T1]{fontenc}
\usepackage{microtype}

\usepackage{amsmath, latexsym, amsfonts, amssymb, amsthm, amscd}
%\usepackage{showkeys}
\usepackage{graphics,epsf,epsfig,psfrag}
\usepackage{perpage}
\usepackage{lastpage}
\usepackage{url}
\usepackage{color}
\usepackage{titlesec}
\usepackage{multicol}
\renewcommand{\multicolsep}{0.1em}

\usepackage[a4paper,scale={0.75,0.75},marginratio={1:1}]{geometry}

%%%%%%%%%%%%%%%%%%%%%%%%%%%%%%%%%%%%%%%%%%%%%%%%%%%%%%%%%%%%%%
%%%%%%%%%%%%%%%%% FANCY PAGE LAYOUT %%%%%%%%%%%%%%%%%%%%%%%%%%
%%%%%%%%%%%%%%%%%%%%%%%%%%%%%%%%%%%%%%%%%%%%%%%%%%%%%%%%%%%%%%

\usepackage{fancyhdr}

\pagestyle{fancy}
%\fancyhead[R]{\sf Curriculum Vit\ae}
%\fancyhead[L]{\sf Francesco Caravenna}
\fancyhead[R]{}
\fancyhead[L]{}
\cfoot{\small \thepage\ / \pageref{LastPage}}
\renewcommand{\headrulewidth}{0pt}
\renewcommand{\footrulewidth}{0pt}
\setlength{\footskip}{8mm}


\fancypagestyle{plain}{% 
%\fancyhead{\sf ~} % empty header
\fancyhf{}
\cfoot{\small \thepage\ / \pageref{LastPage}}
\renewcommand{\headrulewidth}{0pt}
\renewcommand{\footrulewidth}{0pt}
\setlength{\footskip}{6.2mm}
}


\frenchspacing

\setlength{\parindent}{0pt}

%\usepackage[T1]{fontenc}
%%\usepackage[latin1]{inputenc} %under Linux
%\usepackage[ansinew]{inputenc} %under Windows


%%%%%%%%%%%%%%%%%%%%%%%%%%%%%%%%%%%%%%%%%%%%%%%%%%%%%%%%%%%%%%
%%%%%%%%%%%%%%%%% LIST ENVIRONMENTS %%%%%%%%%%%%%%%%%%%%%%%%%%
%%%%%%%%%%%%%%%%%%%%%%%%%%%%%%%%%%%%%%%%%%%%%%%%%%%%%%%%%%%%%%


\newenvironment{myenumerate}{%
\renewcommand{\theenumi}{\arabic{enumi}}%
\renewcommand{\labelenumi}{{\rm(\theenumi)}}%
\begin{list}{\labelenumi}
	{%
	\setlength{\itemsep}{0.3em}%
	\setlength{\topsep}{0.3em}%
	\setlength{\partopsep}{0em}%
	\setlength{\parsep}{0em}%
	\setlength{\parskip}{0em}%
	\setlength\leftmargin{1.5em}%
	\setlength\labelwidth{1.1em}%
	\setlength{\labelsep}{0.4em}%
	\usecounter{enumi}%
	}%
	}%
{\end{list}
}


\renewenvironment{enumerate}{
\begin{myenumerate}}%
{\end{myenumerate}} 


\newenvironment{myitemize}{%
\begin{list}{$\bullet$}% 
 	{%
	\setlength{\itemsep}{0.3em}%
	\setlength{\topsep}{0.3em}%
	\setlength{\partopsep}{0em}%
	\setlength{\parsep}{0em}%
	\setlength{\parskip}{0em}%
	\setlength\leftmargin{1.5em}%
	\setlength\labelwidth{1.1em}%
	\setlength{\labelsep}{0.4em}%
%	\usecounter{enumi}%
	}%
	}%
{\end{list}}


\renewenvironment{itemize}{
\begin{myitemize}}%
{\end{myitemize}} 


%%%%%%%%%%%%%%%%%%%%%%%%%%%%%%%%%%%%%%%%%%%%%%%%%%%%%%%%%%%%%%
%%%%%%%%%%%%%%%%%%%%%%%%% FOOTNOTES %%%%%%%%%%%%%%%%%%%%%%%%%%
%%%%%%%%%%%%%%%%%%%%%%%%%%%%%%%%%%%%%%%%%%%%%%%%%%%%%%%%%%%%%%

\MakePerPage[2]{footnote} % restarts footnote counter at every new page
\def\thefootnote{\fnsymbol{footnote}} % prints footnotes markers as symbols

%\long\def\symbolfootnote[#1]#2{\begingroup\def\thefootnote{\fnsymbol{footnote}}%
%\footnote[#1]{#2}\endgroup}
% for unnumbered footnotes, use \symbolfootnote[0]{text here}
% ... 1=*, 2=dagger, 3=doubledagger, etc.


%%%%%%%%%%%%%%%%%%%%%%%%%%%%%%%%%%%%%%%%%%%%%%%%%%%%%%%%%%%%%%
%%%%%%%%%%%%%%%%%%%%%%%%%%%%%%%%%%%%%%%%%%%%%%%%%%%%%%%%%%%%%%
%%%%%%%%%%%%%%%%%%%%%%%%%%%%%%%%%%%%%%%%%%%%%%%%%%%%%%%%%%%%%%

\definecolor{myred}{RGB}{176,16,16}
\definecolor{myblack}{RGB}{0,0,0}

\titleformat{\section}{\color{myred}\titlerule[0.7pt]\color{myblack}}
{\color{myred}\rule[-0.4em]{0.7pt}{1.55em}\color{myblack}}
{.35em}{\bfseries\sffamily\MakeUppercase}

\titlespacing{\section}{0pt}{*3}{*1.5}
%\titlespacing{\section}{-0.35em}{*3}{*1.5}


%\newcommand{\mysection}[1]{%
%  \ifhmode\par\fi
%  \removelastskip
%  \vskip 0.7em
%  \begingroup
%  \textcolor{myred}{\noindent\rule[-0.5ex]{\columnwidth}{1.2pt}\nobreak}\par
%  \noindent
%  \vskip -0.15em\nobreak
%  \leavevmode\textcolor{myred}{\rule[-0.35em]{1.2pt}{1.55em}}%
%  \MakeUppercase{\hskip 0.8ex \bfseries \sffamily #1}\par
%  \vskip 0.5em\nobreak
%  \endgroup%
%%  \textcolor{myred}{\noindent\hrulefill\nobreak}
%  \nopagebreak%
%  }
%
%\renewcommand{\section}{\mysection}

\def\myskip{\vskip .175\baselineskip plus .175\baselineskip}
%\def\myskip{\smallskip}

%%%%%%%%%%%%%%%%%%%%%%%%%%%%%%%%%%%%%%%%%%%%%%%%%%%%%%%%%%%%%%
%%%%%%%%%%%%%%%%%%%%%%%%%%%%%%%%%%%%%%%%%%%%%%%%%%%%%%%%%%%%%%
%%%%%%%%%%%%%%%%%%%%%%%%%%%%%%%%%%%%%%%%%%%%%%%%%%%%%%%%%%%%%%

\usepackage{pxfonts} % Sets Palatino as main font
\renewcommand{\sfdefault}{iwona} % Sets Iwona as sans-serif font
%\usepackage[euler-digits]{eulervm} % Alternative math fonts (Euler)

\begin{document}
\LARGE
{\huge\sffamily\bfseries Francesco Caravenna}\hfill 
{\normalsize\sffamily \raisebox{0.53em}{[Last updated: \today]}}

{\sffamily Curriculum Vit\ae}

\vspace{1.5ex}

%\begin{center}
%{\sffamily\bfseries Curriculum Vit\ae\ of Francesco Caravenna}
%
%{\large\sffamily [Last updated: \today]}
%\end{center}

\normalsize

%\mbox{}

%\vfill

\section{Personal Details}

\begin{itemize}

\item \emph{Birth:} March 15, 1979, in Treviglio (BG), Italy

%\myskip

\item \emph{Working address:} via Cozzi 55,\ 20125 Milano,\ Italy\\
\phantom{\emph{Working address:}} office 3062, U5 building\\
\phantom{\emph{Working address:}} phone +39 02 6448 5752

%\myskip

%\item \emph{Old working address:} via Trieste 63,\ 35121 Padova,\ Italy\\
%\emph{(still in use)}\phantom{\hskip 3.8em} office 539, ``Torre Archimede'' building\\
%\phantom{\emph{Old working address:}} phone +39 049 827 1472

%\myskip

\item \emph{E-mail:} \texttt{francesco.caravenna@unimib.it}

%\myskip

\item \emph{Home page:} \url{http://www.matapp.unimib.it/~fcaraven/}

%\myskip

\item \emph{Language skills:} Italian (mother tongue), English, French

%\myskip

\item \emph{Computer skills:} C, R, HTML/CSS, \LaTeX
\end{itemize}





\section{Education}

\begin{itemize}
\item
(21 Oct 2005) 
\emph{Ph.D. in Mathematics}, University of Milano-Bicocca (I)
and University of Paris 7~--~Denis Diderot (F). Advisors: Giambattista Giacomin
and Alberto Gandolfi.

\item
(19 Dec 2003)
\emph{`Diploma di Licenza' in Physics} cum laude, %(mark: 70/70 e lode),
Scuola Normale Superiore of Pisa (I). Advisor: Sergio Caracciolo.

\item
(23 Sep 2002)
\emph{Master's Degree in Physics} cum laude, %(mark: 110/110 e lode),
University of Pisa (I). Advisor: Sergio Caracciolo.
\end{itemize}

\section{Academic Positions}

\begin{itemize}
\item (Nov 2010 -- present) \emph{Associate Professor of Mathematics
(Probability and Statistics)},
Department of Mathematics and Applications, University of Milano-Bicocca (I).

\item (Oct 2006 -- Oct 2010) \emph{Assistant Professor (`Ricercatore')
of Mathematics (Probability and Statistics)},
Department of Pure and Applied Mathematics, University of Padova (I).

\item (Oct 2005 -- Sep 2006) \emph{Postdoctoral Fellow in Mathematics}
in the group of Erwin Bolthausen, 
Institute of Mathematics,
University of Z\"urich (CH).
\end{itemize}



\section{Honors}

\begin{itemize}
\item Winner of the ``Guido Fubini Prize 2011'', 
awarded by the Istituto Superiore Mario Boella to a young mathematician working in Italy 
in the field of random processes.
\end{itemize}

\section{Ph.D. Students}

\begin{itemize}
\item Niccol\`o Torri, \emph{``Localization and universality phenomena for
random polymers''}, thesis defended on 18 September 2015,
University of Milano-Bicocca (I) in partnership with the University Claude Bernarde Lyon 1 (F)
[joint supervision with Fabio Lucio Toninelli]

\item Jacopo Corbetta, \emph{``General smile asymptotics and a multiscaling
stochastic volatility model''}, thesis defended on 4 March 2015,
University of Milano-Bicocca (I).
\end{itemize}

\section{Grants}

\begin{itemize}
\item Principal Investigator of the Grant
\emph{`Probabilistic models for the statistical mechanics of polymers, 
interacting particle systems and applications'} (CPDA082105/08) of the University of Padova
[founding: 34\,000 euros]

\item Member of National Research Projects (PRIN) in probability / statistical
mechanics (years 2004, 2006, 2009)
\end{itemize}

\section{Organization Activity}

\begin{itemize}
\item Organization of the Winter School
\emph{``Recent Breakthroughs in Singular Stochastic PDEs''}
(Milano-Bicocca, 2-6 February 2015),
mini-courses by Massimiliano Gubinelli and Lorenzo Zambotti 
[in collaboration with Federica Masiero and Gianmario Tessitore].

\item Member of the organizing committee of the
\emph{``XII Workshop on Quantitative Finance''} (Padova, January 27-28, 2011).
\end{itemize}

\section{Research Visits}

\begin{itemize}
\item Five months at the \emph{Mathematical Institute} of the \emph{University of Leiden} (NL)
as a visiting professor in the framework of the
ERC Advanced Grant VARIS, at the invitation of Frank den Hollander
(17 March - 11 May 2012; 1 May - 30 June 2013; 1-30 March 2014) 

\item Two months at the
\textit{Laboratoire de Probabilit\'es et Mod\`eles Al\'eatoires},
\textit{Universities of Paris 6 and Paris 7} (F), at the invitation of Giambattista Giacomin
and Lorenzo Zambotti  (5 October - 13 December 2009) .

\item One month at the \emph{Institute of Mathematics} of the
\emph{Technical University of Berlin} (D)
at the invitation of Jean-Dominique Deuschel (8 January - 2 February 2007) .

\item Several shorter periods (one / two weeks) at the \emph{University of Warwick} 
(16-21 Jun 2014), \emph{National University of Singapore} (4-15 Apr 2016,
27 Jan - 8 Feb 2014),
\emph{University of Nantes} (18-22 Oct 2010),
\emph{Technical University of Berlin} (3-14 Nov 2008), \emph{ENS Lyon} (16-20 Jun 2008),
\textit{University of Paris 7} (17-21 Jul 2006, 19-23 May 2008),
\textit{University of Z\"urich} (28 Apr - 2 May 2008),
\emph{Technical University of Eindhoven} / \emph{Eurandom}
 (10-14 Jul 2006, 22-26 Oct 2007).

%
%\item
%(16-21 Jun 2014)
%One week at the \emph{University of Warwick} (UK)
%at the invitation of Nikos Zygouras.
%
%\item
%(1-30 Mar 2014)
%One month at the \emph{University of Leiden} (NL)
%at the invitation of Frank den Hollander.
%
%\item
%(27 Jan - 8 Feb 2014)
%Two weeks at the \emph{National University of Singapore}
%at the invitation of Rongfeng Sun.
%
%\item
%(1 May - 30 Jun 2013)
%Two months at the \emph{University of Leiden} (NL)
%at the invitation of Frank den Hollander.
%
%\item
%(17 Mar - 11 May 2012)
%Two months at the \emph{University of Leiden} (NL)
%at the invitation of Frank den Hollander.
%
%\item
%(18-22 Oct 2010)
%One week at the \textit{University of Nantes} (F)
%at the invitation of Nicolas P\'etr\'elis.
%
%\item
%(5 Oct -- 13 Dec 2009)
%Two months at the 
%\emph{Laboratoire de Probabilit\'es et Mod\`eles Al\'eatoires},
%\emph{Universities of Paris 6 and Paris 7} (F), at the
%invitation of Giambattista Giacomin and Lorenzo Zambotti.
%
%\item
%(3-14 Nov 2008) Two weeks at the \emph{Technical University of Berlin} (D)
%at the invitation of Jean-Dominique Deuschel and Nicolas P\'etr\'elis.
%
%\item
%(16-20 Jun 2008) One week at \emph{ENS Lyon} (F) at the invitation of 
%Fabio Toninelli.
%
%\item
%(19-23 May 2008)
%One week at the \emph{University of
%Paris~7} (F) at the invitation of Giam\-bat\-ti\-sta Giacomin.
%
%\item
%(28 Apr - 2 May 2008)
%One week at the \emph{University of Z\"urich} (CH)
%at the invitation of Erwin Bolthausen and Nicolas P\'etr\'elis.
%
%\item
%(22-26 Oct 2007)
%One week at \emph{Eurandom} in
%Eindhoven (NL) at the invitation of Nicolas P\'etr\'elis.
%
%\item
%(5-9 Mar 2007; 23-27 Apr 2007)
%Two weeks at the \emph{University of
%Paris~7} (F) at the invitation of Giambattista Giacomin.
%
%\item
%(8 Jan - 2 Feb 2007) 
%One month at the \emph{Technical University of Berlin} (D)
%at the invitation of Jean-Dominique Deuschel.
%
%\item
%(17-21 Jul 2006)
%One week at the \emph{University of
%Paris~7} (F) at the invitation of Giambattista Giacomin.
%
%\item
%(10-14 Jul 2006)
%One week at \emph{Eurandom} in
%Eindhoven (NL) at the invitation of Nicolas P\'etr\'elis.
\end{itemize}




\section{Referee Activity}

\begin{itemize}
\item I acted as referee for the following journals:
\textit{Probability Theory and Related Fields}, 
\textit{Annals of Probability},
\textit{Communications in Mathematical Physics},
\textit{Stochastic Processes and their Applications},
\textit{Electronic Journal of Probability},
\textit{Annales de l'Institut Henri Poincar\'e}, 
\textit{Statistics and Probability Letters},
\textit{Markov Processes and Related Fields},
\textit{Potential Analysis}.

\end{itemize}

%\pagebreak

\section{Research Interests}

My research activity is focused on probability theory and its applications.
My main research lines are the following:
\begin{enumerate}
\item[\bf 1.] Statistical mechanics, in particular probabilistic models for polymers
\item[\bf 2.] Asymptotic properties of real random walks
\item[\bf 3.] Probabilistic models for financial series
\end{enumerate}

\smallskip
{\hskip 1.2em}I have co-authored 23 papers published on top international journals,
such as \emph{J. Eur. Math. Soc. (JEMS)}, 
\emph{Probab. Theory Related Fields}, \emph{Ann. Probab.}, 
\emph{Commun. Math. Phys.}, 
\emph{Ann. Appl. Probab.}, 
\emph{Stochastic Process. Appl.}, 
\emph{Electron. J. Probab.}, 
\emph{Ann. Inst. H. Poincar\'e}.



\section{Publications}

\medskip
{\large\sf \bfseries Journal papers}
\smallskip

\input{../Publications/pub}

\smallskip
\medskip
{\large\sf \bfseries Proceedings}
\smallskip

\input{../Publications/proc}

\smallskip
\medskip
{\large\sf \bfseries Textbook}
\smallskip

\begin{itemize}
\item F. Caravenna, P. Dai Pra.
\textit{Probabilit\`a. Un'introduzione attraverso modelli e applicazioni.}
Springer-Verlag Italia, Milano (2013).
\end{itemize}


%\textsf{BOOK}
%
%\begin{itemize}
%\item F. Caravenna, P. Dai Pra.
%\textit{Probabilit\`a. Un'introduzione attraverso modelli e applicazioni.}
%%UNITEXT - La matematica per il 3+2.
%Springer-Verlag Italia, Milano (2013).
%\end{itemize}
%
%\smallskip
%
%\newpage

\section{Seminars}

\smallskip
I gave several invited talks at national and international conferences, as well as
many other talks in universities and research centers, plus a few didactic talks.

% EXTENDED
%{\hskip 1.2em}Ho tenuto per due volte (2007 e 2015) una conferenza breve 
%su invito al Congresso dell'Unione
%Matematica Italiana; sono stato \emph{plenary speaker}
%all'Italian Meeting on Hyperbolic Equations (2013);
%ho tenuto il Colloquium del Dipartimento di Matematica 
%all'Universit\`a di Leiden (2014).


%\newpage

%\smallskip
\medskip
{\large\sf \bfseries Invited Talks}
\smallskip


\begin{itemize}
\item (14 Jun 2016)
\emph{Scaling and universality in Probability},
Mathematics Colloquium of the University of Luxembourg (L).

\item (10 Sep 2015)
\emph{A multilinear extension of the central limit theorem
and the universality phenomenon for disordered systems},
XX Congress of the Italian Mathematical Union, Siena (I).

\item (31 Aug 2015)
\emph{Universality in marginally relevant disordered systems},
Meeting on ``Scaling Limits in Models of Statistical Mechanics'',
Oberwolfach Center of Mathematical Research (D).

\item (5 May 2015)
\emph{Polynomial chaos and scaling limits of disordered systems},
Workshop on Stochastic Processes in Random Media,  Institute for Mathematical Sciences,
National University of Singapore.

\item (28 Jul 2014)
\emph{The continuum disordered pinning model},
SPA 2014 Conference, ``Random Polymers'' Invited Session, Buenos Aires (AR).

\item (6 Jun 2014)
\emph{Polynomial chaos and scaling limits of disordered systems},
Statistical Mechanics Conference, University of Nantes (F).

\item (27 Apr 2014)
\emph{Polynomial chaos and scaling limits of disordered systems},
Mini-Workshop at NYU Abu Dhabi (UAE).

\item (13 Mar 2014)
\emph{Scaling and universality in Probability},
General Colloquium, Institute of Mathematics,
University of Leiden (NL).

\item (12 Sep 2013)
\emph{Scaling limits and universality for random pinning models},
15th Italian Meeting on Hyperbolic Equations,
University of Milano-Bicocca (I).

\item (9 Aug 2013)
\emph{Scaling limits and universality for random pinning models},
Workshop ``Universality and Scaling Limits in Probability and Statistical Mechanics'',
Hokkaido University, Sapporo (JP).

\item (21 Mar 2013)
\emph{Scaling limits and universality for random pinning models},
Workshop ``Analysis and Stochastics in Complex Physical Systems'', University of Leipzig (D).

%\item (10-12 Jan 2013)
%\emph{Random Polymers and Localization Strategies} [mini-course],
%School of ``Random Polymers'', Eurandom, Eindhoven~(NL).
%
%\item (14-18 May 2012)
%\emph{Random Polymers and Localization Strategies} [mini-course],
%Worshop ``Random Polymers and Related Topics'', Institute for Mathematical Sciences,
%National University of Singapore.
%
%\item (6-10 Sep 2011)
%\emph{The strange behavior of a random walk} [mini-course],
%Mathematics Summer School ``Alfa Class'' for top undergraduate students
%by ``Progetto Diderot'' -- Fondazione CRT, Solonghello (AL), Italy.

\item (15 Feb 2011)
\emph{The weak coupling limit of disordered copolymer models},
Workshop on Interacting Processes in Random Environments,
Fields Instutute, Toronto (CDN).

\item (16 Oct 2010)
\emph{The weak coupling limit of disordered copolymer models},
Workshop ``Probabilistic Methods in Statistical Physics''
on the occasion of Erwin Bolthausen's
65th birthday, Technische Universit\"at Berlin (D).

%\item (2-7 Aug 2010)
%\emph{Random Polymers} 
%[tutorials for the course given by Frank den Hollander;
%in collaboration with Nicolas P\'etr\'elis],
%XIV Brazilian School of Probability and
%Clay Mathematics Institute 2010 Summer School
%``Probability and Statistical Physics in Two and more Dimensions'',
%B\'uzios -- Rio de Janeiro (BR).

\item (7 Jun 2010)
\emph{Scaling and multiscaling in financial indexes: a simple model},
Workshop ``Statistical Mechanics and Random Media'',
University of Nantes~(F).

\item (11 Jun 2008)
\emph{Pinning and wetting transition for
(1+1)--dimensional fields with Laplacian interaction},
Workshop on Gradient Models and Elasticity, University of Warwick (UK).

\item (6 Mar 2008)
\emph{The quenched critical point of a diluted disordered polymer model},
GREFI-MEFI 2008 Workshop, CIRM, Marseille (F).

\item (26 Sep 2007)
\emph{Polymer models and random walks},
XVIII Congress of the Italian Mathematical Union, Bari (I).

\item (22 Jun 2007)
\emph{On the phase diagram of random copolymers at selective interfaces},
Workshop ``Random Polymer Models'',
Eurandom, Eindhoven~(NL).

\item (21 Feb 2007)
\emph{Pinning and wetting models with Laplacian interaction in (1+1)--di\-men\-sion},
Workshop ``Polymer Models and Related Topics'',
Laboratoire J.A. Dieudonn\'e, University of Nice `Sophia Antipolis'~(F).

\item (4 Sep 2006)
\emph{Pinning models with Laplacian interactions in (1+1)--dimension},
Meeting on ``Spatial Random Processes and Statistical Mechanics'',
Oberwolfach Center of Mathematical Research (D).

\item (6 Jun 2006)
\emph{A renewal theory approach to weakly inhomogeneous polymer models},
Workshop ``Hydrodynamic Limits and Particle Systems'',
\emph{Ennio De Giorgi} Center of Mathematical Research in Pisa (I).
\end{itemize}


\pagebreak

\medskip
{\large\sf \bfseries Other Talks}
\smallskip

\begin{itemize}
\item (15 Jun 2016)
\emph{Universality in marginally relevant disordered systems},
University of Luxembourg (L).

\item (2 Jun 2016)
\emph{Universality in marginally relevant disordered systems},
University of Paris Diderot (F).

\item (19 Nov 2015)
\emph{Multi-linear central limit theorems and scaling limits of disordered systems},
Vienna Seminar in Mathematical Finance and Probability (A).

\item (19 Mar 2015)
\emph{Polynomial chaos and scaling limits of disordered systems},
Probability Seminar, ENS Lyon (F).

\item (11 Oct 2013)
\emph{Polynomial chaos and scaling limits of disordered systems},
University of Padova (I).

\item (12 Jul 2013)
\emph{Scaling limits and universality for random pinning models},
Rhein-Main Kolloquium Stochastik, University of Mainz (D).

\item (13 Jun 2013)
\emph{Scaling limits and universality for random pinning models},
Most Informal Probability Seminar, University of Leiden (NL).

\item (10 Jun 2013)
\emph{Scaling limits and universality for random pinning models},
University of Angers (F).

\item (27 Jul 2012)
\emph{Scaling and multiscaling in financial series: a simple model},
Department of Statistics and Quantitative Methods,
University of Milano-Bicocca (I).

\item (10 Jul 2012)
\emph{A random copolymer near a selective interface},
University of Roma Sapienza (I).

\item (3 Apr 2012)
\emph{Bootstrap percolation on Galton Watson trees},
Most Informal Probability Seminar, University of Leiden (NL).

\item (12 Mar 2012)
\emph{Scaling and multiscaling in financial series: a simple model},
University of Modena and Reggio Emilia (I).

\item (23 Sep 2011)
\emph{Scaling and multiscaling in financial indexes: a simple model},
University of Rome Tor Vergata (I).

\item (16 Dec 2010)
\emph{A polymer in a multi-interface medium},
Oberseminar Stochastics,
University of Bonn (D).

\item (2 Nov 2010)
\emph{Scaling and multiscaling in financial indexes: a simple model},
University of Milano-Bicocca (I).

\item (10 Jun 2010)
\emph{The weak coupling limit of disordered copolymer models},
University of Warwick~(UK).

\item (20 May 2010)
\emph{Scaling and multiscaling in financial indexes: a simple model},
University of Padova~(I).

\item (1 Feb 2010)
\emph{Scaling and multiscaling in financial indexes: a simple model},
University of Pavia~(I).

\item (8 Dec 2009)
\emph{A polymer in a multi-interface medium},
S\'eminaire de Probabilit\'es,
Laboratoire de Probabilit\'es et Mod\`eles Al\'eatoires,
Universities of Paris 6 and Paris 7 (F).

\item (5 Nov 2009)
\emph{Large scale behavior of semiflexible heteropolymers},
University of Nantes (F).

\item (21 Jul 2009)
\emph{A polymer in a multi-interface medium},
University of Roma 3 (I).

\item (12 Nov 2008)
\emph{A polymer in a multi-interface medium},
Berliner Kolloquium Wahr\-schein\-lich\-keits\-theo\-rie
(IRTG Seminar), Humboldt University in Berlin (D).

\item (19 Jun 2008)
\emph{Pinning and wetting transition for
(1+1)--dimensional fields with Laplacian interaction},
S\'eminaire de Probabilit\'es, ENS Lyon (F).

\item (30 Apr 2008)
\emph{Pinning and wetting transition for
(1+1)--dimensional fields with Laplacian interaction},
Seminar on Stochastic Processes, University of Z\"urich (CH).

\item (23 Oct 2007)
\emph{The quenched critical point of a diluted disordered polymer model},
Random Spatial Structures Seminar, Eurandom, Eindhoven (NL).

\item (17 Jul 2007)
\emph{The quenched critical point of a diluted disordered polymer model},
St.~Flour Summer School of Probability (F).

\item (24 Jan 2007)
\emph{Free energy lower bounds for random copolymers at selective interfaces},
Technical University of Berlin~(D).

\item (13 Jul 2006)
\emph{Pinning models with Laplacian interactions in (1+1)--dimension},
Random Spatial Structures Seminar, Eurandom, Eindhoven (NL).

\item (17 Mar 2006)
\emph{A renewal theory approach to periodically inhomogeneous polymer models},
Seminaire de Probabilit\'es et Statistique, Centre de Math\'ematiques et Informatique,
University of Provence, Marseille (F).

\item (21 Dec 2005)
\emph{A local limit theorem and an invariance principle for random walks conditioned
to stay positive},
Seminar on Stochastic Processes, ETH Z\"urich (CH).

\item (16 Dec 2005)
\emph{A renewal theory approach to periodically inhomogeneous polymer models},
Berlin-Leipzig Seminar, Technical University of Berlin (D).

\item (19 Jul 2005)
\emph{A renewal theory approach to periodically inhomogeneous polymer models},
St.~Flour Summer School of Probability (F).

\item (27 May 2005)
\emph{Random copolymers at selective interfaces},
University of Milano-Bicocca (I).

\item (2 Jun 2004)
\emph{A local limit theorem for random walks conditioned to stay positive},
Rencontres de Probabilit\'es,
University of Rouen (F).
\end{itemize}

\medskip
{\large\sf \bfseries Didactic Talks}
\smallskip

\begin{itemize}
\item (12 Mar 2015) \emph{Mathematics in the university},
talk for an audience of high school students, Open Day
``Privavera in Bicocca 2015''.

\item (14 Mar 2013) \emph{Research in mathematics},
talk for an audience of high school students, Open Day
``Privavera in Bicocca 2013''.

\item (5 Nov 2008) \emph{The Banach-Tarski Paradox},
Technische Universit\"at Berlin (D).

\item (29 June 2006) \emph{What is\ldots the Banach-Tarski Paradox?},
Z\"urich Graduate Colloquium (CH).
\end{itemize}


%\pagebreak

\section{Teaching Experience}

\nopagebreak

Since my Ph.D., I have given many undergraduate courses and recitations in several
universities (Milano-Bicocca, Padova, Zurich, Politecnico di Milano).
I have also given more advanced courses and mini-courses for Ph.D. programs, schools
and workshops.

\medskip
{\large\sf \bfseries Courses and mini-courses in schools and workshops}
\smallskip

\begin{itemize}
\item (2 Jun 2016)
\emph{Polynomial chaos and scaling limits of disordered systems} [3h],
%[corso ($5 \times 1.5$ ore)],
Working group on ``Statistical mechanics and particle systems'',
University of Paris Diderot (F).

\item (7 - 11 Mar 2016)
\emph{Polynomial chaos and scaling limits of disordered systems}
[3~x~1.5h],
Workshop YEP XIII, Eurandom, TU Eindhoven (NL).

\item (28 Sep - 2 Oct 2015)
\emph{Polynomial chaos and scaling limits of disordered systems}
[5~x~1.5h],
Berlin-Potsdam Summer School 2015, Levico Terme (I).

\item (10-12 Jan 2013)
\emph{Random Polymers and Localization Strategies} [3~x~2h],
School of ``Random Polymers'', Eurandom, Eindhoven~(NL).

\item (14-18 May 2012)
\emph{Random Polymers and Localization Strategies} [3~x~2h + 1h],
Worshop ``Random Polymers and Related Topics'', Institute for Mathematical Sciences,
National University of Singapore.

\item (6-10 Sep 2011)
\emph{The strange behavior of a random walk} [1.5h + 2h],
Mathematics Summer School ``Alfa Class'' for top undergraduate students
by ``Progetto Diderot'' -- Fondazione CRT, Solonghello (AL), Italy.

\item (2-7 Aug 2010)
\emph{Random Polymers} 
[5x1h tutorials, in collaboration with Nicolas P\'etr\'elis,
for the course given by Frank den Hollander],
XIV Brazilian School of Probability and
Clay Mathematics Institute 2010 Summer School
``Probability and Statistical Physics in Two and more Dimensions'',
B\'uzios -- Rio de Janeiro (BR).
\end{itemize}


\medskip
{\large\sf \bfseries Ph.D.\ Courses}
\smallskip



\begin{itemize}
\item (Spring 2014)
\textit{Random Graphs and Complex Networks},
Ph.D. in Mathematics, joint course ("interdottorato") for the Universities of
Milano, Milano-Bicocca, Pavia and Politecnico of Milano
[in collaboration with Federico Bassetti].

\item (Spring 2010) 
\textit{Random Graphs and Complex Networks},
Ph.D. in Mathematics, University of Padova (I)
[in collaboration with Paolo Dai Pra].

\item (Spring 2009)
\emph{Poisson Point Processes and Applications},
Ph.D. in Mathematics, University of Padova (I).

\item (Fall 2006)
\emph{Polymer Models and Random Walks},
Ph.D. in Mathematics, University of Padova (I).
\end{itemize}


\medskip
{\large\sf \bfseries Undergraduate Courses and Recitations}
\smallskip

\emph{At the University of Milano-Bicocca}

\begin{itemize}
\item (Spring 2016)
\textit{Stochastic Processes} (course),
Master's Degree in Mathematics
[in collaboration with Gianmario Tessitore].

\item (Fall 2015)
\textit{Probability Theory} (course),
Bachelor's Degree in Mathematics.

\item (Spring 2015)
\textit{Stochastic Processes} (course),
Master's Degree in Mathematics
[in collaboration with Gianmario Tessitore].

\item (Fall 2014)
\textit{Probability Theory} (course + recitations),
Bachelor's Degree in Mathematics.

\item (Spring 2014)
\textit{Stochastic Processes} (course),
Master's Degree in Mathematics
[in collaboration with Gianmario Tessitore].

\item (Fall 2013)
\textit{Probability Theory} (course + recitations),
Bachelor's Degree in Mathematics.

\item (Spring 2013)
\textit{Stochastic Processes} (course),
Master's Degree in Mathematics
[in collaboration with Gianmario Tessitore].

%\item (Spring 2013)
%\emph{Random Polymers and Localization Strategies} (mini-course),
%School of ``Random Polymers'', Eurandom, Eindhoven~(NL),
%10-12 Jan 2013.

\item (Fall 2012)
\textit{Probability Theory} (course + recitations),
Bachelor's Degree in Mathematics.

%\item (Spring 2012)
%\emph{Random Polymers and Localization Strategies} (mini-course),
%Worshop ``Random Polymers and Related Topics'', Institute for Mathematical Sciences,
%National University of Singapore, 14-18 May 2012.

\item (Spring 2012)
\textit{Stochastic Processes} (course),
Master's Degree in Mathematics
[in collaboration with Gianmario Tessitore].

\item (Fall 2011)
\textit{Probability Theory} (course + recitations),
Bachelor's Degree in Mathematics.

\item (Spring 2011)
\textit{Stochastic Processes} (course),
Master's Degree in Mathematics
[in collaboration with Gianmario Tessitore].

\item (Spring 2005)
\emph{Stochastic Processes} (recitations),
Master's Degree in Mathematics 
[teacher: Gianmario Tessitore].

\item (Spring 2004)
\emph{Stochastic Processes} (recitations),
Master's Degree in Mathematics 
[teacher: Daniela Bertacchi].

\item (Spring 2004)
\emph{Probability and Statistics} (recitations),
Bachelor's Degree in Mathematics 
[teacher: Daniela Bertacchi].
\end{itemize}

\pagebreak

\smallskip
\emph{At the University of Padova}

\begin{itemize}
\item (Spring 2011)
\emph{Stochastic Calculus} (course),
Master's Degree in Mathematics
[in collaboration with David Barbato].

%\item (2-7 Aug 2010)
%Tutorials for the mini-course \emph{Random Polymers},
%XIV Brazilian School of Probability and
%Clay Mathematics Institute 2010 Summer School
%``Probability and Statistical Physics in Two and more Dimensions'',
%B\'uzios -- Rio de Janeiro (BR)
%[teacher: Frank den Hollander; in collaboration with Nicolas P\'etr\'elis].

\item (Spring 2010)
\emph{Statistics} (recitations),
Bachelor's Degree in Molecular Biology
[teacher: Paolo Dai Pra].

\item (Spring 2010)
\emph{Stochastic Calculus} (course),
Master's Degree in Mathematics.

\item (Spring 2009)
\emph{Probability and Statistics} (recitations),
Bachelor's Degree in Mathematics
[teacher: Paolo Dai Pra].

\item (Spring 2009)
\emph{Stochastic Calculus} (course),
Master's Degree in Mathematics.

\item (Spring 2008)
\emph{Probability Theory} (course),
Bachelor's Degree in Engineering, 
University of Padova (I) [in collaboration with Caterina Sartori].

\item (Spring 2008)
\emph{Probability and Statistics} (recitations),
Bachelor's Degree in Mathematics
[teacher: Paolo Dai Pra].

\item (Spring 2008)
\emph{Statistical Methods for Biology} (recitations),
Bachelor's Degree in Biology
[teacher: Paolo Dai Pra].

\item (Spring 2007)
\emph{Statistical Methods for Biology} (recitations),
Bachelor's Degree in Molecular Biology
[teacher: Paolo Dai Pra].

\item (Spring 2007)
\emph{Statistical Methods for Biology} (recitations),
Bachelor's Degree in Biology
[teacher: Paolo Dai Pra].
\end{itemize}

\smallskip
\emph{At the University of Zurich}


\begin{itemize}
\item (Spring 2006)
\emph{Linear Algebra II} (recitations),
Bachelor's Degree in Mathematics
[teacher: Erwin Bolthausen].

\item (Fall 2005)
\emph{Linear Algebra I} (recitations),
Bachelor's Degree in Mathematics
[teacher: Erwin Bolthausen].
\end{itemize}

\smallskip
\emph{At Politecnico di Milano}

\begin{itemize}
\item (Spring 2003)
\emph{Probability and Statistics} (on-line recitations),
Bachelor's Degree in Engineering
[teacher: Elio Piazza].

\item (Spring 2003)
\emph{Probability and Statistics} (numerical laboratory),
Bachelor's Degree in Engineering [teacher: Elio Piazza].
\end{itemize}


\end{document}
